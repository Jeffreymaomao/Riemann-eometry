\documentclass[11pt]{article}

    \usepackage[breakable]{tcolorbox}
    \usepackage{parskip} % Stop auto-indenting (to mimic markdown behaviour)
    \usepackage{pmboxdraw}
    

    % Basic figure setup, for now with no caption control since it's done
    % automatically by Pandoc (which extracts ![](path) syntax from Markdown).
    \usepackage{graphicx}
    % Maintain compatibility with old templates. Remove in nbconvert 6.0
    \let\Oldincludegraphics\includegraphics
    % Ensure that by default, figures have no caption (until we provide a
    % proper Figure object with a Caption API and a way to capture that
    % in the conversion process - todo).
    \usepackage{caption}
    \DeclareCaptionFormat{nocaption}{}
    \captionsetup{format=nocaption,aboveskip=0pt,belowskip=0pt}

    \usepackage{float}
    \floatplacement{figure}{H} % forces figures to be placed at the correct location
    \usepackage{xcolor} % Allow colors to be defined
    \usepackage{enumerate} % Needed for markdown enumerations to work
    \usepackage{geometry} % Used to adjust the document margins
    \usepackage{amsmath} % Equations
    \usepackage{amssymb} % Equations
    \usepackage{textcomp} % defines textquotesingle
    % Hack from http://tex.stackexchange.com/a/47451/13684:
    \AtBeginDocument{%
        \def\PYZsq{\textquotesingle}% Upright quotes in Pygmentized code
    }
    \usepackage{upquote} % Upright quotes for verbatim code
    \usepackage{eurosym} % defines \euro
	
	\usepackage{amssymb}
	\let\mathbbalt\mathbb
	
    \usepackage{iftex}
    \ifPDFTeX
        \usepackage[T1]{fontenc}
        \IfFileExists{alphabeta.sty}{
              \usepackage{alphabeta}
          }{
              \usepackage[mathletters]{ucs}
              \usepackage[utf8x]{inputenc}
          }
    \else
        \usepackage{fontspec}
        \usepackage{unicode-math}
    \fi
    \let\mathbb\mathbbalt

    \usepackage{fancyvrb} % verbatim replacement that allows latex
    \usepackage{grffile} % extends the file name processing of package graphics
                         % to support a larger range
    \makeatletter % fix for old versions of grffile with XeLaTeX
    \@ifpackagelater{grffile}{2019/11/01}
    {
      % Do nothing on new versions
    }
    {
      \def\Gread@@xetex#1{%
        \IfFileExists{"\Gin@base".bb}%
        {\Gread@eps{\Gin@base.bb}}%
        {\Gread@@xetex@aux#1}%
      }
    }
    \makeatother
    \usepackage[Export]{adjustbox} % Used to constrain images to a maximum size
    \adjustboxset{max size={0.9\linewidth}{0.9\paperheight}}

    % The hyperref package gives us a pdf with properly built
    % internal navigation ('pdf bookmarks' for the table of contents,
    % internal cross-reference links, web links for URLs, etc.)
    \usepackage{hyperref}
    % The default LaTeX title has an obnoxious amount of whitespace. By default,
    % titling removes some of it. It also provides customization options.
    \usepackage{titling}
    \usepackage{longtable} % longtable support required by pandoc >1.10
    \usepackage{booktabs}  % table support for pandoc > 1.12.2
    \usepackage{array}     % table support for pandoc >= 2.11.3
    \usepackage{calc}      % table minipage width calculation for pandoc >= 2.11.1
    \usepackage[inline]{enumitem} % IRkernel/repr support (it uses the enumerate* environment)
    \usepackage[normalem]{ulem} % ulem is needed to support strikethroughs (\sout)
                                % normalem makes italics be italics, not underlines
    \usepackage{soul}      % strikethrough (\st) support for pandoc >= 3.0.0
    \usepackage{mathrsfs}
    

    
    % Colors for the hyperref package
    \definecolor{urlcolor}{rgb}{0,.145,.698}
    \definecolor{linkcolor}{rgb}{.71,0.21,0.01}
    \definecolor{citecolor}{rgb}{.12,.54,.11}

    % ANSI colors
    \definecolor{ansi-black}{HTML}{3E424D}
    \definecolor{ansi-black-intense}{HTML}{282C36}
    \definecolor{ansi-red}{HTML}{E75C58}
    \definecolor{ansi-red-intense}{HTML}{B22B31}
    \definecolor{ansi-green}{HTML}{00A250}
    \definecolor{ansi-green-intense}{HTML}{007427}
    \definecolor{ansi-yellow}{HTML}{DDB62B}
    \definecolor{ansi-yellow-intense}{HTML}{B27D12}
    \definecolor{ansi-blue}{HTML}{208FFB}
    \definecolor{ansi-blue-intense}{HTML}{0065CA}
    \definecolor{ansi-magenta}{HTML}{D160C4}
    \definecolor{ansi-magenta-intense}{HTML}{A03196}
    \definecolor{ansi-cyan}{HTML}{60C6C8}
    \definecolor{ansi-cyan-intense}{HTML}{258F8F}
    \definecolor{ansi-white}{HTML}{C5C1B4}
    \definecolor{ansi-white-intense}{HTML}{A1A6B2}
    \definecolor{ansi-default-inverse-fg}{HTML}{FFFFFF}
    \definecolor{ansi-default-inverse-bg}{HTML}{000000}

    % common color for the border for error outputs.
    \definecolor{outerrorbackground}{HTML}{FFDFDF}

    % commands and environments needed by pandoc snippets
    % extracted from the output of `pandoc -s`
    \providecommand{\tightlist}{%
      \setlength{\itemsep}{0pt}\setlength{\parskip}{0pt}}
    \DefineVerbatimEnvironment{Highlighting}{Verbatim}{commandchars=\\\{\}}
    % Add ',fontsize=\small' for more characters per line
    \newenvironment{Shaded}{}{}
    \newcommand{\KeywordTok}[1]{\textcolor[rgb]{0.00,0.44,0.13}{\textbf{{#1}}}}
    \newcommand{\DataTypeTok}[1]{\textcolor[rgb]{0.56,0.13,0.00}{{#1}}}
    \newcommand{\DecValTok}[1]{\textcolor[rgb]{0.25,0.63,0.44}{{#1}}}
    \newcommand{\BaseNTok}[1]{\textcolor[rgb]{0.25,0.63,0.44}{{#1}}}
    \newcommand{\FloatTok}[1]{\textcolor[rgb]{0.25,0.63,0.44}{{#1}}}
    \newcommand{\CharTok}[1]{\textcolor[rgb]{0.25,0.44,0.63}{{#1}}}
    \newcommand{\StringTok}[1]{\textcolor[rgb]{0.25,0.44,0.63}{{#1}}}
    \newcommand{\CommentTok}[1]{\textcolor[rgb]{0.38,0.63,0.69}{\textit{{#1}}}}
    \newcommand{\OtherTok}[1]{\textcolor[rgb]{0.00,0.44,0.13}{{#1}}}
    \newcommand{\AlertTok}[1]{\textcolor[rgb]{1.00,0.00,0.00}{\textbf{{#1}}}}
    \newcommand{\FunctionTok}[1]{\textcolor[rgb]{0.02,0.16,0.49}{{#1}}}
    \newcommand{\RegionMarkerTok}[1]{{#1}}
    \newcommand{\ErrorTok}[1]{\textcolor[rgb]{1.00,0.00,0.00}{\textbf{{#1}}}}
    \newcommand{\NormalTok}[1]{{#1}}

    % Additional commands for more recent versions of Pandoc
    \newcommand{\ConstantTok}[1]{\textcolor[rgb]{0.53,0.00,0.00}{{#1}}}
    \newcommand{\SpecialCharTok}[1]{\textcolor[rgb]{0.25,0.44,0.63}{{#1}}}
    \newcommand{\VerbatimStringTok}[1]{\textcolor[rgb]{0.25,0.44,0.63}{{#1}}}
    \newcommand{\SpecialStringTok}[1]{\textcolor[rgb]{0.73,0.40,0.53}{{#1}}}
    \newcommand{\ImportTok}[1]{{#1}}
    \newcommand{\DocumentationTok}[1]{\textcolor[rgb]{0.73,0.13,0.13}{\textit{{#1}}}}
    \newcommand{\AnnotationTok}[1]{\textcolor[rgb]{0.38,0.63,0.69}{\textbf{\textit{{#1}}}}}
    \newcommand{\CommentVarTok}[1]{\textcolor[rgb]{0.38,0.63,0.69}{\textbf{\textit{{#1}}}}}
    \newcommand{\VariableTok}[1]{\textcolor[rgb]{0.10,0.09,0.49}{{#1}}}
    \newcommand{\ControlFlowTok}[1]{\textcolor[rgb]{0.00,0.44,0.13}{\textbf{{#1}}}}
    \newcommand{\OperatorTok}[1]{\textcolor[rgb]{0.40,0.40,0.40}{{#1}}}
    \newcommand{\BuiltInTok}[1]{{#1}}
    \newcommand{\ExtensionTok}[1]{{#1}}
    \newcommand{\PreprocessorTok}[1]{\textcolor[rgb]{0.74,0.48,0.00}{{#1}}}
    \newcommand{\AttributeTok}[1]{\textcolor[rgb]{0.49,0.56,0.16}{{#1}}}
    \newcommand{\InformationTok}[1]{\textcolor[rgb]{0.38,0.63,0.69}{\textbf{\textit{{#1}}}}}
    \newcommand{\WarningTok}[1]{\textcolor[rgb]{0.38,0.63,0.69}{\textbf{\textit{{#1}}}}}


    % Define a nice break command that doesn't care if a line doesn't already
    % exist.
    \def\br{\hspace*{\fill} \\* }
    % Math Jax compatibility definitions
    \def\gt{>}
    \def\lt{<}
    \let\Oldtex\TeX
    \let\Oldlatex\LaTeX
    \renewcommand{\TeX}{\textrm{\Oldtex}}
    \renewcommand{\LaTeX}{\textrm{\Oldlatex}}
    % Document parameters
    % Document title
    \title{Riemann Geometry}
    \date{\today}
    \author{Chang-Mao, Yang}
    
    
    
    
    
    
    
% Pygments definitions
\makeatletter
\def\PY@reset{\let\PY@it=\relax \let\PY@bf=\relax%
    \let\PY@ul=\relax \let\PY@tc=\relax%
    \let\PY@bc=\relax \let\PY@ff=\relax}
\def\PY@tok#1{\csname PY@tok@#1\endcsname}
\def\PY@toks#1+{\ifx\relax#1\empty\else%
    \PY@tok{#1}\expandafter\PY@toks\fi}
\def\PY@do#1{\PY@bc{\PY@tc{\PY@ul{%
    \PY@it{\PY@bf{\PY@ff{#1}}}}}}}
\def\PY#1#2{\PY@reset\PY@toks#1+\relax+\PY@do{#2}}

\@namedef{PY@tok@w}{\def\PY@tc##1{\textcolor[rgb]{0.73,0.73,0.73}{##1}}}
\@namedef{PY@tok@c}{\let\PY@it=\textit\def\PY@tc##1{\textcolor[rgb]{0.24,0.48,0.48}{##1}}}
\@namedef{PY@tok@cp}{\def\PY@tc##1{\textcolor[rgb]{0.61,0.40,0.00}{##1}}}
\@namedef{PY@tok@k}{\let\PY@bf=\textbf\def\PY@tc##1{\textcolor[rgb]{0.00,0.50,0.00}{##1}}}
\@namedef{PY@tok@kp}{\def\PY@tc##1{\textcolor[rgb]{0.00,0.50,0.00}{##1}}}
\@namedef{PY@tok@kt}{\def\PY@tc##1{\textcolor[rgb]{0.69,0.00,0.25}{##1}}}
\@namedef{PY@tok@o}{\def\PY@tc##1{\textcolor[rgb]{0.40,0.40,0.40}{##1}}}
\@namedef{PY@tok@ow}{\let\PY@bf=\textbf\def\PY@tc##1{\textcolor[rgb]{0.67,0.13,1.00}{##1}}}
\@namedef{PY@tok@nb}{\def\PY@tc##1{\textcolor[rgb]{0.00,0.50,0.00}{##1}}}
\@namedef{PY@tok@nf}{\def\PY@tc##1{\textcolor[rgb]{0.00,0.00,1.00}{##1}}}
\@namedef{PY@tok@nc}{\let\PY@bf=\textbf\def\PY@tc##1{\textcolor[rgb]{0.00,0.00,1.00}{##1}}}
\@namedef{PY@tok@nn}{\let\PY@bf=\textbf\def\PY@tc##1{\textcolor[rgb]{0.00,0.00,1.00}{##1}}}
\@namedef{PY@tok@ne}{\let\PY@bf=\textbf\def\PY@tc##1{\textcolor[rgb]{0.80,0.25,0.22}{##1}}}
\@namedef{PY@tok@nv}{\def\PY@tc##1{\textcolor[rgb]{0.10,0.09,0.49}{##1}}}
\@namedef{PY@tok@no}{\def\PY@tc##1{\textcolor[rgb]{0.53,0.00,0.00}{##1}}}
\@namedef{PY@tok@nl}{\def\PY@tc##1{\textcolor[rgb]{0.46,0.46,0.00}{##1}}}
\@namedef{PY@tok@ni}{\let\PY@bf=\textbf\def\PY@tc##1{\textcolor[rgb]{0.44,0.44,0.44}{##1}}}
\@namedef{PY@tok@na}{\def\PY@tc##1{\textcolor[rgb]{0.41,0.47,0.13}{##1}}}
\@namedef{PY@tok@nt}{\let\PY@bf=\textbf\def\PY@tc##1{\textcolor[rgb]{0.00,0.50,0.00}{##1}}}
\@namedef{PY@tok@nd}{\def\PY@tc##1{\textcolor[rgb]{0.67,0.13,1.00}{##1}}}
\@namedef{PY@tok@s}{\def\PY@tc##1{\textcolor[rgb]{0.73,0.13,0.13}{##1}}}
\@namedef{PY@tok@sd}{\let\PY@it=\textit\def\PY@tc##1{\textcolor[rgb]{0.73,0.13,0.13}{##1}}}
\@namedef{PY@tok@si}{\let\PY@bf=\textbf\def\PY@tc##1{\textcolor[rgb]{0.64,0.35,0.47}{##1}}}
\@namedef{PY@tok@se}{\let\PY@bf=\textbf\def\PY@tc##1{\textcolor[rgb]{0.67,0.36,0.12}{##1}}}
\@namedef{PY@tok@sr}{\def\PY@tc##1{\textcolor[rgb]{0.64,0.35,0.47}{##1}}}
\@namedef{PY@tok@ss}{\def\PY@tc##1{\textcolor[rgb]{0.10,0.09,0.49}{##1}}}
\@namedef{PY@tok@sx}{\def\PY@tc##1{\textcolor[rgb]{0.00,0.50,0.00}{##1}}}
\@namedef{PY@tok@m}{\def\PY@tc##1{\textcolor[rgb]{0.40,0.40,0.40}{##1}}}
\@namedef{PY@tok@gh}{\let\PY@bf=\textbf\def\PY@tc##1{\textcolor[rgb]{0.00,0.00,0.50}{##1}}}
\@namedef{PY@tok@gu}{\let\PY@bf=\textbf\def\PY@tc##1{\textcolor[rgb]{0.50,0.00,0.50}{##1}}}
\@namedef{PY@tok@gd}{\def\PY@tc##1{\textcolor[rgb]{0.63,0.00,0.00}{##1}}}
\@namedef{PY@tok@gi}{\def\PY@tc##1{\textcolor[rgb]{0.00,0.52,0.00}{##1}}}
\@namedef{PY@tok@gr}{\def\PY@tc##1{\textcolor[rgb]{0.89,0.00,0.00}{##1}}}
\@namedef{PY@tok@ge}{\let\PY@it=\textit}
\@namedef{PY@tok@gs}{\let\PY@bf=\textbf}
\@namedef{PY@tok@gp}{\let\PY@bf=\textbf\def\PY@tc##1{\textcolor[rgb]{0.00,0.00,0.50}{##1}}}
\@namedef{PY@tok@go}{\def\PY@tc##1{\textcolor[rgb]{0.44,0.44,0.44}{##1}}}
\@namedef{PY@tok@gt}{\def\PY@tc##1{\textcolor[rgb]{0.00,0.27,0.87}{##1}}}
\@namedef{PY@tok@err}{\def\PY@bc##1{{\setlength{\fboxsep}{\string -\fboxrule}\fcolorbox[rgb]{1.00,0.00,0.00}{1,1,1}{\strut ##1}}}}
\@namedef{PY@tok@kc}{\let\PY@bf=\textbf\def\PY@tc##1{\textcolor[rgb]{0.00,0.50,0.00}{##1}}}
\@namedef{PY@tok@kd}{\let\PY@bf=\textbf\def\PY@tc##1{\textcolor[rgb]{0.00,0.50,0.00}{##1}}}
\@namedef{PY@tok@kn}{\let\PY@bf=\textbf\def\PY@tc##1{\textcolor[rgb]{0.00,0.50,0.00}{##1}}}
\@namedef{PY@tok@kr}{\let\PY@bf=\textbf\def\PY@tc##1{\textcolor[rgb]{0.00,0.50,0.00}{##1}}}
\@namedef{PY@tok@bp}{\def\PY@tc##1{\textcolor[rgb]{0.00,0.50,0.00}{##1}}}
\@namedef{PY@tok@fm}{\def\PY@tc##1{\textcolor[rgb]{0.00,0.00,1.00}{##1}}}
\@namedef{PY@tok@vc}{\def\PY@tc##1{\textcolor[rgb]{0.10,0.09,0.49}{##1}}}
\@namedef{PY@tok@vg}{\def\PY@tc##1{\textcolor[rgb]{0.10,0.09,0.49}{##1}}}
\@namedef{PY@tok@vi}{\def\PY@tc##1{\textcolor[rgb]{0.10,0.09,0.49}{##1}}}
\@namedef{PY@tok@vm}{\def\PY@tc##1{\textcolor[rgb]{0.10,0.09,0.49}{##1}}}
\@namedef{PY@tok@sa}{\def\PY@tc##1{\textcolor[rgb]{0.73,0.13,0.13}{##1}}}
\@namedef{PY@tok@sb}{\def\PY@tc##1{\textcolor[rgb]{0.73,0.13,0.13}{##1}}}
\@namedef{PY@tok@sc}{\def\PY@tc##1{\textcolor[rgb]{0.73,0.13,0.13}{##1}}}
\@namedef{PY@tok@dl}{\def\PY@tc##1{\textcolor[rgb]{0.73,0.13,0.13}{##1}}}
\@namedef{PY@tok@s2}{\def\PY@tc##1{\textcolor[rgb]{0.73,0.13,0.13}{##1}}}
\@namedef{PY@tok@sh}{\def\PY@tc##1{\textcolor[rgb]{0.73,0.13,0.13}{##1}}}
\@namedef{PY@tok@s1}{\def\PY@tc##1{\textcolor[rgb]{0.73,0.13,0.13}{##1}}}
\@namedef{PY@tok@mb}{\def\PY@tc##1{\textcolor[rgb]{0.40,0.40,0.40}{##1}}}
\@namedef{PY@tok@mf}{\def\PY@tc##1{\textcolor[rgb]{0.40,0.40,0.40}{##1}}}
\@namedef{PY@tok@mh}{\def\PY@tc##1{\textcolor[rgb]{0.40,0.40,0.40}{##1}}}
\@namedef{PY@tok@mi}{\def\PY@tc##1{\textcolor[rgb]{0.40,0.40,0.40}{##1}}}
\@namedef{PY@tok@il}{\def\PY@tc##1{\textcolor[rgb]{0.40,0.40,0.40}{##1}}}
\@namedef{PY@tok@mo}{\def\PY@tc##1{\textcolor[rgb]{0.40,0.40,0.40}{##1}}}
\@namedef{PY@tok@ch}{\let\PY@it=\textit\def\PY@tc##1{\textcolor[rgb]{0.24,0.48,0.48}{##1}}}
\@namedef{PY@tok@cm}{\let\PY@it=\textit\def\PY@tc##1{\textcolor[rgb]{0.24,0.48,0.48}{##1}}}
\@namedef{PY@tok@cpf}{\let\PY@it=\textit\def\PY@tc##1{\textcolor[rgb]{0.24,0.48,0.48}{##1}}}
\@namedef{PY@tok@c1}{\let\PY@it=\textit\def\PY@tc##1{\textcolor[rgb]{0.24,0.48,0.48}{##1}}}
\@namedef{PY@tok@cs}{\let\PY@it=\textit\def\PY@tc##1{\textcolor[rgb]{0.24,0.48,0.48}{##1}}}

\def\PYZbs{\char`\\}
\def\PYZus{\char`\_}
\def\PYZob{\char`\{}
\def\PYZcb{\char`\}}
\def\PYZca{\char`\^}
\def\PYZam{\char`\&}
\def\PYZlt{\char`\<}
\def\PYZgt{\char`\>}
\def\PYZsh{\char`\#}
\def\PYZpc{\char`\%}
\def\PYZdl{\char`\$}
\def\PYZhy{\char`\-}
\def\PYZsq{\char`\'}
\def\PYZdq{\char`\"}
\def\PYZti{\char`\~}
% for compatibility with earlier versions
\def\PYZat{@}
\def\PYZlb{[}
\def\PYZrb{]}
\makeatother


    % For linebreaks inside Verbatim environment from package fancyvrb.
    \makeatletter
        \newbox\Wrappedcontinuationbox
        \newbox\Wrappedvisiblespacebox
        \newcommand*\Wrappedvisiblespace {\textcolor{red}{\textvisiblespace}}
        \newcommand*\Wrappedcontinuationsymbol {\textcolor{red}{\llap{\tiny$\m@th\hookrightarrow$}}}
        \newcommand*\Wrappedcontinuationindent {3ex }
        \newcommand*\Wrappedafterbreak {\kern\Wrappedcontinuationindent\copy\Wrappedcontinuationbox}
        % Take advantage of the already applied Pygments mark-up to insert
        % potential linebreaks for TeX processing.
        %        {, <, #, %, $, ' and ": go to next line.
        %        _, }, ^, &, >, - and ~: stay at end of broken line.
        % Use of \textquotesingle for straight quote.
        \newcommand*\Wrappedbreaksatspecials {%
            \def\PYGZus{\discretionary{\char`\_}{\Wrappedafterbreak}{\char`\_}}%
            \def\PYGZob{\discretionary{}{\Wrappedafterbreak\char`\{}{\char`\{}}%
            \def\PYGZcb{\discretionary{\char`\}}{\Wrappedafterbreak}{\char`\}}}%
            \def\PYGZca{\discretionary{\char`\^}{\Wrappedafterbreak}{\char`\^}}%
            \def\PYGZam{\discretionary{\char`\&}{\Wrappedafterbreak}{\char`\&}}%
            \def\PYGZlt{\discretionary{}{\Wrappedafterbreak\char`\<}{\char`\<}}%
            \def\PYGZgt{\discretionary{\char`\>}{\Wrappedafterbreak}{\char`\>}}%
            \def\PYGZsh{\discretionary{}{\Wrappedafterbreak\char`\#}{\char`\#}}%
            \def\PYGZpc{\discretionary{}{\Wrappedafterbreak\char`\%}{\char`\%}}%
            \def\PYGZdl{\discretionary{}{\Wrappedafterbreak\char`\$}{\char`\$}}%
            \def\PYGZhy{\discretionary{\char`\-}{\Wrappedafterbreak}{\char`\-}}%
            \def\PYGZsq{\discretionary{}{\Wrappedafterbreak\textquotesingle}{\textquotesingle}}%
            \def\PYGZdq{\discretionary{}{\Wrappedafterbreak\char`\"}{\char`\"}}%
            \def\PYGZti{\discretionary{\char`\~}{\Wrappedafterbreak}{\char`\~}}%
        }
        % Some characters . , ; ? ! / are not pygmentized.
        % This macro makes them "active" and they will insert potential linebreaks
        \newcommand*\Wrappedbreaksatpunct {%
            \lccode`\~`\.\lowercase{\def~}{\discretionary{\hbox{\char`\.}}{\Wrappedafterbreak}{\hbox{\char`\.}}}%
            \lccode`\~`\,\lowercase{\def~}{\discretionary{\hbox{\char`\,}}{\Wrappedafterbreak}{\hbox{\char`\,}}}%
            \lccode`\~`\;\lowercase{\def~}{\discretionary{\hbox{\char`\;}}{\Wrappedafterbreak}{\hbox{\char`\;}}}%
            \lccode`\~`\:\lowercase{\def~}{\discretionary{\hbox{\char`\:}}{\Wrappedafterbreak}{\hbox{\char`\:}}}%
            \lccode`\~`\?\lowercase{\def~}{\discretionary{\hbox{\char`\?}}{\Wrappedafterbreak}{\hbox{\char`\?}}}%
            \lccode`\~`\!\lowercase{\def~}{\discretionary{\hbox{\char`\!}}{\Wrappedafterbreak}{\hbox{\char`\!}}}%
            \lccode`\~`\/\lowercase{\def~}{\discretionary{\hbox{\char`\/}}{\Wrappedafterbreak}{\hbox{\char`\/}}}%
            \catcode`\.\active
            \catcode`\,\active
            \catcode`\;\active
            \catcode`\:\active
            \catcode`\?\active
            \catcode`\!\active
            \catcode`\/\active
            \lccode`\~`\~
        }
    \makeatother

    \let\OriginalVerbatim=\Verbatim
    \makeatletter
    \renewcommand{\Verbatim}[1][1]{%
        %\parskip\z@skip
        \sbox\Wrappedcontinuationbox {\Wrappedcontinuationsymbol}%
        \sbox\Wrappedvisiblespacebox {\FV@SetupFont\Wrappedvisiblespace}%
        \def\FancyVerbFormatLine ##1{\hsize\linewidth
            \vtop{\raggedright\hyphenpenalty\z@\exhyphenpenalty\z@
                \doublehyphendemerits\z@\finalhyphendemerits\z@
                \strut ##1\strut}%
        }%
        % If the linebreak is at a space, the latter will be displayed as visible
        % space at end of first line, and a continuation symbol starts next line.
        % Stretch/shrink are however usually zero for typewriter font.
        \def\FV@Space {%
            \nobreak\hskip\z@ plus\fontdimen3\font minus\fontdimen4\font
            \discretionary{\copy\Wrappedvisiblespacebox}{\Wrappedafterbreak}
            {\kern\fontdimen2\font}%
        }%

        % Allow breaks at special characters using \PYG... macros.
        \Wrappedbreaksatspecials
        % Breaks at punctuation characters . , ; ? ! and / need catcode=\active
        \OriginalVerbatim[#1,codes*=\Wrappedbreaksatpunct]%
    }
    \makeatother

    % Exact colors from NB
    \definecolor{incolor}{HTML}{303F9F}
    \definecolor{outcolor}{HTML}{D84315}
    \definecolor{cellborder}{HTML}{CFCFCF}
    \definecolor{cellbackground}{HTML}{F7F7F7}

    % prompt
    \makeatletter
    \newcommand{\boxspacing}{\kern\kvtcb@left@rule\kern\kvtcb@boxsep}
    \makeatother
    \newcommand{\prompt}[4]{
        {\ttfamily\llap{{\color{#2}[#3]:\hspace{3pt}#4}}\vspace{-\baselineskip}}
    }
    

    
    % Prevent overflowing lines due to hard-to-break entities
    \sloppy
    % Setup hyperref package
    \hypersetup{
      breaklinks=true,  % so long urls are correctly broken across lines
      colorlinks=true,
      urlcolor=urlcolor,
      linkcolor=linkcolor,
      citecolor=citecolor,
      }
    % Slightly bigger margins than the latex defaults
    
    \geometry{a4paper, margin=1in, paperwidth=16in, paperheight=35in}
	\usepackage{lscape}
	
    
    

\begin{document}	
	
    \maketitle
    
    \tableofcontents	    
    

    
    \hypertarget{riemann-geometry}{%
\section{Riemann Geometry}\label{riemann-geometry}}

    \begin{tcolorbox}[breakable, size=fbox, boxrule=1pt, pad at break*=1mm,colback=cellbackground, colframe=cellborder]
\prompt{In}{incolor}{1}{\boxspacing}
\begin{Verbatim}[commandchars=\\\{\}]
\PY{k+kn}{import} \PY{n+nn}{sympy} \PY{k}{as} \PY{n+nn}{sp}
\PY{k+kn}{from} \PY{n+nn}{IPython}\PY{n+nn}{.}\PY{n+nn}{display} \PY{k+kn}{import} \PY{n}{Math}\PY{p}{,} \PY{n}{Markdown}
\end{Verbatim}
\end{tcolorbox}

    \begin{tcolorbox}[breakable, size=fbox, boxrule=1pt, pad at break*=1mm,colback=cellbackground, colframe=cellborder]
\prompt{In}{incolor}{2}{\boxspacing}
\begin{Verbatim}[commandchars=\\\{\}]
\PY{o}{\PYZpc{}}\PY{k}{reload\PYZus{}ext} autoreload
\PY{o}{\PYZpc{}}\PY{k}{autoreload} 2
\PY{k+kn}{from} \PY{n+nn}{riemann\PYZus{}geometry} \PY{k+kn}{import} \PY{n}{RiemannGeometry} 
\end{Verbatim}
\end{tcolorbox}

    \begin{tcolorbox}[breakable, size=fbox, boxrule=1pt, pad at break*=1mm,colback=cellbackground, colframe=cellborder]
\prompt{In}{incolor}{3}{\boxspacing}
\begin{Verbatim}[commandchars=\\\{\}]
\PY{n}{RiemannGeometry}
\end{Verbatim}
\end{tcolorbox}

            \begin{tcolorbox}[breakable, size=fbox, boxrule=.5pt, pad at break*=1mm, opacityfill=0]
\prompt{Out}{outcolor}{3}{\boxspacing}
\begin{Verbatim}[commandchars=\\\{\}]
riemann\_geometry.core.RiemannGeometry
\end{Verbatim}
\end{tcolorbox}
        
        
    \hypertarget{convension}{%
\subsection{Convension}\label{convension}}

\hypertarget{einstein-convension}{%
\subsubsection{Einstein Convension}\label{einstein-convension}}

Here using the Einstein summation convention, 
\begin{quote}
When an index variable appears twice in a single term and is not otherwise
defined, it implies summation of that term over all the
values of the index.
\end{quote}

\hypertarget{tensor-denotion}{%
\subsubsection{Tensor denotion}\label{tensor-denotion}}

On the manofold \(\mathcal{M}\) with dimension \(\dim\mathcal{M} = n\).
The \textbf{\emph{metric}} \(g_{ab}\) and the \textbf{\emph{inverse of
metric}} \(g^{ab}\) is denoted in matrix form, that is \[
g_{ab} = \begin{pmatrix}
 g_{11} &  g_{12} & \cdots &  g_{1n} \\ 
 g_{21} &  g_{22} & \cdots &  g_{2n} \\ 
 \vdots & \vdots & \ddots &\vdots \\ 
 g_{n1} &  g_{n2} & \cdots &  g_{nn} \\ 
\end{pmatrix}
,\quad 
g^{ab} = \begin{pmatrix}
 g^{11} &  g^{12} & \cdots &  g^{1n} \\ 
 g^{21} &  g^{22} & \cdots &  g^{2n} \\ 
 \vdots & \vdots & \ddots &\vdots \\ 
 g^{n1} &  g^{n2} & \cdots &  g^{nn} \\ 
\end{pmatrix}.
\]

For the \textbf{\emph{Christoffel symbols}} \({\Gamma^{a}}_{bc}\),
notice that it only has \(3\) indices, so here define it in the form \[
{\Gamma^{a}}_{bcd} = 
\begin{pmatrix}
\begin{pmatrix}
 {\Gamma^{1}_{11}} &  {\Gamma^{1}_{12}} & \cdots &  {\Gamma^{1}_{1n}} \\ 
 {\Gamma^{1}_{21}} &  {\Gamma^{1}_{22}} & \cdots &  {\Gamma^{1}_{2n}} \\ 
 \vdots & \vdots & \ddots &\vdots \\ 
 {\Gamma^{1}_{n1}} &  {\Gamma^{1}_{n2}} & \cdots &  {\Gamma^{1}_{nn}} \\ 
\end{pmatrix}
& 
\begin{pmatrix}
 {\Gamma^{2}_{11}} &  {\Gamma^{2}_{12}} & \cdots &  {\Gamma^{2}_{1n}} \\ 
 {\Gamma^{2}_{21}} &  {\Gamma^{2}_{22}} & \cdots &  {\Gamma^{2}_{2n}} \\ 
 \vdots & \vdots & \ddots &\vdots \\ 
 {\Gamma^{2}_{n1}} &  {\Gamma^{2}_{n2}} & \cdots &  {\Gamma^{2}_{nn}} \\ 
\end{pmatrix}
& \cdots & 
\begin{pmatrix}
 {\Gamma^{n}_{11}} &  {\Gamma^{n}_{12}} & \cdots &  {\Gamma^{n}_{1n}} \\ 
 {\Gamma^{n}_{21}} &  {\Gamma^{n}_{22}} & \cdots &  {\Gamma^{n}_{2n}} \\ 
 \vdots & \vdots & \ddots &\vdots \\ 
 {\Gamma^{n}_{n1}} &  {\Gamma^{n}_{n2}} & \cdots &  {\Gamma^{n}_{nn}} \\ 
\end{pmatrix}
\end{pmatrix}
\]

Then the \textbf{\emph{Riemann tensor}} \({R^{a}}_{bcd}\) also denote as
a matrix in a matrix form, \[
{R^{a}}_{bcd} = \begin{pmatrix}
 {R^{1}}_{1cd} & {R^{1}}_{1cd} & \cdots &  {R^{1}}_{1cd} \\ 
 {R^{2}}_{2cd} & {R^{2}}_{2cd} & \cdots &  {R^{2}}_{2cd} \\ 
 \vdots & \vdots & \ddots &\vdots \\ 
 {R^{n}}_{ncd} & {R^{n}}_{ncd} & \cdots &  {R^{n}}_{ncd} \\ 
\end{pmatrix}
= \begin{pmatrix}
\begin{pmatrix}
{R^{1}}_{111} & {R^{1}}_{211} & \cdots & {R^{1}}_{n11} \\ 
{R^{2}}_{111} & {R^{2}}_{211} & \cdots & {R^{2}}_{n11} \\ 
 \vdots & \vdots & \ddots &\vdots \\ 
{R^{n}}_{111} & {R^{n}}_{211} & \cdots & {R^{n}}_{n11} \\ 
\end{pmatrix} & \begin{pmatrix}
{R^{1}}_{112} & {R^{1}}_{212} & \cdots & {R^{1}}_{n12} \\ 
{R^{2}}_{112} & {R^{2}}_{212} & \cdots & {R^{2}}_{n12} \\ 
 \vdots & \vdots & \ddots &\vdots \\ 
{R^{n}}_{112} & {R^{n}}_{212} & \cdots & {R^{n}}_{n12} \\ 
\end{pmatrix} & \cdots & \begin{pmatrix}
{R^{1}}_{11n} & {R^{1}}_{21n} & \cdots & {R^{1}}_{n1n} \\ 
{R^{2}}_{11n} & {R^{2}}_{21n} & \cdots & {R^{2}}_{n1n} \\ 
 \vdots & \vdots & \ddots &\vdots \\ 
{R^{n}}_{11n} & {R^{n}}_{21n} & \cdots & {R^{n}}_{n1n} \\ 
\end{pmatrix} \\ 
\begin{pmatrix}
{R^{1}}_{121} & {R^{1}}_{221} & \cdots & {R^{1}}_{n21} \\ 
{R^{2}}_{121} & {R^{2}}_{221} & \cdots & {R^{2}}_{n21} \\ 
 \vdots & \vdots & \ddots &\vdots \\ 
{R^{n}}_{121} & {R^{n}}_{221} & \cdots & {R^{n}}_{n21} \\ 
\end{pmatrix} & \begin{pmatrix}
{R^{1}}_{122} & {R^{1}}_{222} & \cdots & {R^{1}}_{n22} \\ 
{R^{2}}_{122} & {R^{2}}_{222} & \cdots & {R^{2}}_{n22} \\ 
 \vdots & \vdots & \ddots &\vdots \\ 
{R^{n}}_{122} & {R^{n}}_{222} & \cdots & {R^{n}}_{n22} \\ 
\end{pmatrix} & \cdots & \begin{pmatrix}
{R^{1}}_{12n} & {R^{1}}_{22n} & \cdots & {R^{1}}_{n2n} \\ 
{R^{2}}_{12n} & {R^{2}}_{22n} & \cdots & {R^{2}}_{n2n} \\ 
 \vdots & \vdots & \ddots &\vdots \\ 
{R^{n}}_{12n} & {R^{n}}_{22n} & \cdots & {R^{n}}_{n2n} \\ 
\end{pmatrix} \\ 
 \vdots & \vdots & \ddots &\vdots \\ 
\begin{pmatrix}
{R^{1}}_{1n1} & {R^{1}}_{2n1} & \cdots & {R^{1}}_{nn1} \\ 
{R^{2}}_{1n1} & {R^{2}}_{2n1} & \cdots & {R^{2}}_{nn1} \\ 
 \vdots & \vdots & \ddots &\vdots \\ 
{R^{n}}_{1n1} & {R^{n}}_{2n1} & \cdots & {R^{n}}_{nn1} \\ 
\end{pmatrix} & \begin{pmatrix}
{R^{1}}_{1n2} & {R^{1}}_{2n2} & \cdots & {R^{1}}_{nn2} \\ 
{R^{2}}_{1n2} & {R^{2}}_{2n2} & \cdots & {R^{2}}_{nn2} \\ 
 \vdots & \vdots & \ddots &\vdots \\ 
{R^{n}}_{1n2} & {R^{n}}_{2n2} & \cdots & {R^{n}}_{nn2} \\ 
\end{pmatrix} & \cdots & \begin{pmatrix}
{R^{1}}_{1nn} & {R^{1}}_{2nn} & \cdots & {R^{1}}_{nnn} \\ 
{R^{2}}_{1nn} & {R^{2}}_{2nn} & \cdots & {R^{2}}_{nnn} \\ 
 \vdots & \vdots & \ddots &\vdots \\ 
{R^{n}}_{1nn} & {R^{n}}_{2nn} & \cdots & {R^{n}}_{nnn} \\ 
\end{pmatrix} \\ 
\end{pmatrix}.
\]

Also, for the \textbf{\emph{Ricci tensor}} \(R_{ab}\), it will also
denote as a matrix form (since it only has two indices), that is \[
R_{ab} = {R^{c}}_{acb} = \begin{pmatrix}
 R_{11} & R_{11} & \cdots &  R_{11} \\ 
 R_{22} & R_{22} & \cdots &  R_{22} \\ 
 \vdots & \vdots & \ddots &\vdots \\ 
 R_{nn} & R_{nn} & \cdots &  R_{nn} \\ 
\end{pmatrix}.
\]

Last, since the \textbf{\emph{Ricci scalar}} \(R\) is just a scalar, so
it has no metrix definition.

    \hypertarget{example-for-manifolds-in-euclidean-mathbbrn}{%
\subsection{\texorpdfstring{Example for Manifolds in Euclidean
\(\mathbb{R}^n\)}{Example for Manifolds in Euclidean \textbackslash mathbb\{R\}\^{}n}}\label{example-for-manifolds-in-euclidean-mathbbrn}}

    \hypertarget{flat-manifold-in-mathbbr2}{%
\subsubsection{\texorpdfstring{Flat Manifold in
\(\mathbb{R}^2\)}{Flat Manifold in \textbackslash mathbb\{R\}\^{}2}}\label{flat-manifold-in-mathbbr2}}

    \hypertarget{cartesian-coordinate}{%
\paragraph{Cartesian coordinate}\label{cartesian-coordinate}}

In \(2\)-dimensional flat space \(\mathbb{R}^2\), the line element in
Cartesian coordinate is \[
ds^2 = dx^2 + dy^2 +dz^2 
\] then we can calculate the \textbf{\emph{Christoffel symbols}},
\textbf{\emph{Riemann tensor}}, \textbf{\emph{Ricci tensor}} and
\textbf{\emph{Ricci scalar}}.

    \begin{tcolorbox}[breakable, size=fbox, boxrule=1pt, pad at break*=1mm,colback=cellbackground, colframe=cellborder]
\prompt{In}{incolor}{4}{\boxspacing}
\begin{Verbatim}[commandchars=\\\{\}]
\PY{n}{x}\PY{p}{,} \PY{n}{y} \PY{o}{=} \PY{n}{sp}\PY{o}{.}\PY{n}{symbols}\PY{p}{(}\PY{l+s+s1}{\PYZsq{}}\PY{l+s+s1}{x y}\PY{l+s+s1}{\PYZsq{}}\PY{p}{)}
\PY{n}{coords} \PY{o}{=} \PY{n}{sp}\PY{o}{.}\PY{n}{Matrix}\PY{p}{(}\PY{p}{[}\PY{n}{x}\PY{p}{,} \PY{n}{y}\PY{p}{]}\PY{p}{)}
\PY{n}{metric} \PY{o}{=} \PY{n}{sp}\PY{o}{.}\PY{n}{diag}\PY{p}{(}\PY{l+m+mi}{1}\PY{p}{,} \PY{l+m+mi}{1}\PY{p}{)}
\PY{n}{geo} \PY{o}{=} \PY{n}{RiemannGeometry}\PY{p}{(}\PY{n}{metric}\PY{p}{,} \PY{n}{coords}\PY{p}{)}
\PY{n}{geo}\PY{o}{.}\PY{n}{calculate}\PY{p}{(}\PY{p}{)}
\PY{n}{geo}
\end{Verbatim}
\end{tcolorbox}

    \begin{Verbatim}[commandchars=\\\{\}]
Calculating {\ldots}
    \end{Verbatim}

    \begin{Verbatim}[commandchars=\\\{\}]
Christoffel Symbols : 100\%|\textblock\textblock\textblock\textblock\textblock\textblock\textblock\textblock\textblock\textblock\textblock\textblock\textblock\textblock\textblock\textblock\textblock\textblock\textblock\textblock\textblock\textblock\textblock\textblock\textblock\textblock\textblock\textblock\textblock\textblock\textblock\textblock\textblock| 2/2 [00:00<00:00,
1008.00it/s]
Riemann Tensor      : 100\%|\textblock\textblock\textblock\textblock\textblock\textblock\textblock\textblock\textblock\textblock\textblock\textblock\textblock\textblock\textblock\textblock\textblock\textblock\textblock\textblock\textblock\textblock\textblock\textblock\textblock\textblock\textblock\textblock\textblock\textblock\textblock\textblock\textblock| 2/2 [00:00<00:00,
1059.97it/s]
Ricci Tensor        : 100\%|\textblock\textblock\textblock\textblock\textblock\textblock\textblock\textblock\textblock\textblock\textblock\textblock\textblock\textblock\textblock\textblock\textblock\textblock\textblock\textblock\textblock\textblock\textblock\textblock\textblock\textblock\textblock\textblock\textblock\textblock\textblock\textblock\textblock| 2/2 [00:00<00:00,
6172.63it/s]
Ricci Scalar        : 100\%|\textblock\textblock\textblock\textblock\textblock\textblock\textblock\textblock\textblock\textblock\textblock\textblock\textblock\textblock\textblock\textblock\textblock\textblock\textblock\textblock\textblock\textblock\textblock\textblock\textblock\textblock\textblock\textblock\textblock\textblock\textblock\textblock| 2/2 [00:00<00:00,
50231.19it/s]
    \end{Verbatim}
 
            
\prompt{Out}{outcolor}{4}{}
    
    $$x^{a}=\begin{pmatrix}x\\y\end{pmatrix},\quad g_{ab}=\begin{pmatrix} 1 & 0\\ 0 & 1 \end{pmatrix},\quad g^{ab}=\begin{pmatrix} 1 & 0\\ 0 & 1 \end{pmatrix},$$
$${\Gamma^{a}}_{bc}=\begin{pmatrix} \begin{pmatrix} 0 & 0\\ 0 & 0 \end{pmatrix} & \begin{pmatrix} 0 & 0\\ 0 & 0 \end{pmatrix} \end{pmatrix},$$
$${R^{a}}_{bcd}=\begin{pmatrix} \begin{pmatrix} 0 & 0\\ 0 & 0 \end{pmatrix} & \begin{pmatrix} 0 & 0\\ 0 & 0 \end{pmatrix}\\ \begin{pmatrix} 0 & 0\\ 0 & 0 \end{pmatrix} & \begin{pmatrix} 0 & 0\\ 0 & 0 \end{pmatrix} \end{pmatrix},$$
$$R_{ab}=\begin{pmatrix} 0 & 0\\ 0 & 0 \end{pmatrix},\quad R=0.$$

    \hypertarget{polar-coordinate}{%
\paragraph{Polar coordinate}\label{polar-coordinate}}

In \(2\)-dimensional flat space \(\mathbb{R}^2\), the line element in
polar coordinate is \[
ds^2 = dr^2 + r^2d\theta^2
\] then we can calculate the \textbf{\emph{Christoffel symbols}},
\textbf{\emph{Riemann tensor}}, \textbf{\emph{Ricci tensor}} and
\textbf{\emph{Ricci scalar}}.

    \begin{tcolorbox}[breakable, size=fbox, boxrule=1pt, pad at break*=1mm,colback=cellbackground, colframe=cellborder]
\prompt{In}{incolor}{5}{\boxspacing}
\begin{Verbatim}[commandchars=\\\{\}]
\PY{n}{r}\PY{p}{,} \PY{n}{theta} \PY{o}{=} \PY{n}{sp}\PY{o}{.}\PY{n}{symbols}\PY{p}{(}\PY{l+s+s1}{\PYZsq{}}\PY{l+s+s1}{r theta}\PY{l+s+s1}{\PYZsq{}}\PY{p}{)}
\PY{n}{coords} \PY{o}{=} \PY{n}{sp}\PY{o}{.}\PY{n}{Matrix}\PY{p}{(}\PY{p}{[}\PY{n}{r}\PY{p}{,} \PY{n}{theta}\PY{p}{]}\PY{p}{)}
\PY{n}{metric} \PY{o}{=} \PY{n}{sp}\PY{o}{.}\PY{n}{diag}\PY{p}{(}\PY{l+m+mi}{1}\PY{p}{,} \PY{n}{r}\PY{o}{*}\PY{o}{*}\PY{l+m+mi}{2}\PY{p}{)}
\PY{n}{geo} \PY{o}{=} \PY{n}{RiemannGeometry}\PY{p}{(}\PY{n}{metric}\PY{p}{,} \PY{n}{coords}\PY{p}{)}
\PY{n}{geo}\PY{o}{.}\PY{n}{calculate}\PY{p}{(}\PY{p}{)}
\PY{n}{geo}
\end{Verbatim}
\end{tcolorbox}

    \begin{Verbatim}[commandchars=\\\{\}]
Calculating {\ldots}
    \end{Verbatim}

    \begin{Verbatim}[commandchars=\\\{\}]
Christoffel Symbols : 100\%|\textblock\textblock\textblock\textblock\textblock\textblock\textblock\textblock\textblock\textblock\textblock\textblock\textblock\textblock\textblock\textblock\textblock\textblock\textblock\textblock\textblock\textblock\textblock\textblock\textblock\textblock\textblock\textblock\textblock\textblock\textblock\textblock\textblock\textblock| 2/2 [00:00<00:00,
308.02it/s]
Riemann Tensor      : 100\%|\textblock\textblock\textblock\textblock\textblock\textblock\textblock\textblock\textblock\textblock\textblock\textblock\textblock\textblock\textblock\textblock\textblock\textblock\textblock\textblock\textblock\textblock\textblock\textblock\textblock\textblock\textblock\textblock\textblock\textblock\textblock\textblock\textblock\textblock| 2/2 [00:00<00:00,
613.34it/s]
Ricci Tensor        : 100\%|\textblock\textblock\textblock\textblock\textblock\textblock\textblock\textblock\textblock\textblock\textblock\textblock\textblock\textblock\textblock\textblock\textblock\textblock\textblock\textblock\textblock\textblock\textblock\textblock\textblock\textblock\textblock\textblock\textblock\textblock\textblock\textblock\textblock| 2/2 [00:00<00:00,
6355.01it/s]
Ricci Scalar        : 100\%|\textblock\textblock\textblock\textblock\textblock\textblock\textblock\textblock\textblock\textblock\textblock\textblock\textblock\textblock\textblock\textblock\textblock\textblock\textblock\textblock\textblock\textblock\textblock\textblock\textblock\textblock\textblock\textblock\textblock\textblock\textblock\textblock| 2/2 [00:00<00:00,
21959.71it/s]
    \end{Verbatim}
 
            
\prompt{Out}{outcolor}{5}{}
    
    $$x^{a}=\begin{pmatrix}r\\\theta\end{pmatrix},\quad g_{ab}=\begin{pmatrix} 1 & 0\\ 0 & r^{2} \end{pmatrix},\quad g^{ab}=\begin{pmatrix} 1 & 0\\ 0 & \frac{1}{r^{2}} \end{pmatrix},$$
$${\Gamma^{a}}_{bc}=\begin{pmatrix} \begin{pmatrix} 0 & 0\\ 0 & - r \end{pmatrix} & \begin{pmatrix} 0 & \frac{1}{r}\\ \frac{1}{r} & 0 \end{pmatrix} \end{pmatrix},$$
$${R^{a}}_{bcd}=\begin{pmatrix} \begin{pmatrix} 0 & 0\\ 0 & 0 \end{pmatrix} & \begin{pmatrix} 0 & 0\\ 0 & 0 \end{pmatrix}\\ \begin{pmatrix} 0 & 0\\ 0 & 0 \end{pmatrix} & \begin{pmatrix} 0 & 0\\ 0 & 0 \end{pmatrix} \end{pmatrix},$$
$$R_{ab}=\begin{pmatrix} 0 & 0\\ 0 & 0 \end{pmatrix},\quad R=0.$$

    
\newpage %%%%%%%%%%%%%%%%%%%%%%%%%%%%%%%%%%%%%%%%%%%%%%%%%%%%%%%%


    \hypertarget{flat-manifold-in-mathbbr3}{%
\subsubsection{\texorpdfstring{Flat Manifold in
\(\mathbb{R}^3\)}{Flat Manifold in \textbackslash mathbb\{R\}\^{}3}}\label{flat-manifold-in-mathbbr3}}

    \hypertarget{cartesian-coordinate}{%
\paragraph{Cartesian coordinate}\label{cartesian-coordinate}}

In \(3\)-dimensional flat space \(\mathbb{R}^3\), the line element in
Cartesian coordinate is \[
ds^2 = dx^2 + dy^2 +dz^2 
\] then we can calculate the \textbf{\emph{Christoffel symbols}},
\textbf{\emph{Riemann tensor}}, \textbf{\emph{Ricci tensor}} and
\textbf{\emph{Ricci scalar}}.

    \begin{tcolorbox}[breakable, size=fbox, boxrule=1pt, pad at break*=1mm,colback=cellbackground, colframe=cellborder]
\prompt{In}{incolor}{6}{\boxspacing}
\begin{Verbatim}[commandchars=\\\{\}]
\PY{n}{x}\PY{p}{,} \PY{n}{y}\PY{p}{,} \PY{n}{z} \PY{o}{=} \PY{n}{sp}\PY{o}{.}\PY{n}{symbols}\PY{p}{(}\PY{l+s+s1}{\PYZsq{}}\PY{l+s+s1}{x y z}\PY{l+s+s1}{\PYZsq{}}\PY{p}{)}
\PY{n}{coords} \PY{o}{=} \PY{n}{sp}\PY{o}{.}\PY{n}{Matrix}\PY{p}{(}\PY{p}{[}\PY{n}{x}\PY{p}{,} \PY{n}{y}\PY{p}{,} \PY{n}{z}\PY{p}{]}\PY{p}{)}
\PY{n}{metric} \PY{o}{=} \PY{n}{sp}\PY{o}{.}\PY{n}{diag}\PY{p}{(}\PY{l+m+mi}{1}\PY{p}{,} \PY{l+m+mi}{1}\PY{p}{,} \PY{l+m+mi}{1}\PY{p}{)}
\PY{n}{geo} \PY{o}{=} \PY{n}{RiemannGeometry}\PY{p}{(}\PY{n}{metric}\PY{p}{,} \PY{n}{coords}\PY{p}{)}
\PY{n}{geo}\PY{o}{.}\PY{n}{calculate}\PY{p}{(}\PY{p}{)}
\PY{n}{geo}
\end{Verbatim}
\end{tcolorbox}

    \begin{Verbatim}[commandchars=\\\{\}]
Calculating {\ldots}
    \end{Verbatim}

    \begin{Verbatim}[commandchars=\\\{\}]
Christoffel Symbols : 100\%|\textblock\textblock\textblock\textblock\textblock\textblock\textblock\textblock\textblock\textblock\textblock\textblock\textblock\textblock\textblock\textblock\textblock\textblock\textblock\textblock\textblock\textblock\textblock\textblock\textblock\textblock\textblock\textblock\textblock\textblock\textblock\textblock\textblock\textblock| 3/3 [00:00<00:00,576.67it/s]
Riemann Tensor      : 100\%|\textblock\textblock\textblock\textblock\textblock\textblock\textblock\textblock\textblock\textblock\textblock\textblock\textblock\textblock\textblock\textblock\textblock\textblock\textblock\textblock\textblock\textblock\textblock\textblock\textblock\textblock\textblock\textblock\textblock\textblock\textblock\textblock\textblock\textblock| 3/3 [00:00<00:00,361.92it/s]
Ricci Tensor        : 100\%|\textblock\textblock\textblock\textblock\textblock\textblock\textblock\textblock\textblock\textblock\textblock\textblock\textblock\textblock\textblock\textblock\textblock\textblock\textblock\textblock\textblock\textblock\textblock\textblock\textblock\textblock\textblock\textblock\textblock\textblock\textblock\textblock\textblock| 3/3 [00:00<00:00,5007.13it/s]
Ricci Scalar        : 100\%|\textblock\textblock\textblock\textblock\textblock\textblock\textblock\textblock\textblock\textblock\textblock\textblock\textblock\textblock\textblock\textblock\textblock\textblock\textblock\textblock\textblock\textblock\textblock\textblock\textblock\textblock\textblock\textblock\textblock\textblock\textblock\textblock| 3/3 [00:00<00:00,17050.02it/s]
    \end{Verbatim}
 
            
\prompt{Out}{outcolor}{6}{}

    $$x^{a}=\begin{pmatrix}x\\y\\z\end{pmatrix},\quad g_{ab}=\begin{pmatrix} 1 & 0 & 0\\ 0 & 1 & 0\\ 0 & 0 & 1 \end{pmatrix},\quad g^{ab}=\begin{pmatrix} 1 & 0 & 0\\ 0 & 1 & 0\\ 0 & 0 & 1 \end{pmatrix},$$
$${\Gamma^{a}}_{bc}=\begin{pmatrix} \begin{pmatrix} 0 & 0 & 0\\ 0 & 0 & 0\\ 0 & 0 & 0 \end{pmatrix} & \begin{pmatrix} 0 & 0 & 0\\ 0 & 0 & 0\\ 0 & 0 & 0 \end{pmatrix} & \begin{pmatrix} 0 & 0 & 0\\ 0 & 0 & 0\\ 0 & 0 & 0 \end{pmatrix} \end{pmatrix},$$
$${R^{a}}_{bcd}=\begin{pmatrix} \begin{pmatrix} 0 & 0 & 0\\ 0 & 0 & 0\\ 0 & 0 & 0 \end{pmatrix} & \begin{pmatrix} 0 & 0 & 0\\ 0 & 0 & 0\\ 0 & 0 & 0 \end{pmatrix} & \begin{pmatrix} 0 & 0 & 0\\ 0 & 0 & 0\\ 0 & 0 & 0 \end{pmatrix}\\ \begin{pmatrix} 0 & 0 & 0\\ 0 & 0 & 0\\ 0 & 0 & 0 \end{pmatrix} & \begin{pmatrix} 0 & 0 & 0\\ 0 & 0 & 0\\ 0 & 0 & 0 \end{pmatrix} & \begin{pmatrix} 0 & 0 & 0\\ 0 & 0 & 0\\ 0 & 0 & 0 \end{pmatrix}\\ \begin{pmatrix} 0 & 0 & 0\\ 0 & 0 & 0\\ 0 & 0 & 0 \end{pmatrix} & \begin{pmatrix} 0 & 0 & 0\\ 0 & 0 & 0\\ 0 & 0 & 0 \end{pmatrix} & \begin{pmatrix} 0 & 0 & 0\\ 0 & 0 & 0\\ 0 & 0 & 0 \end{pmatrix} \end{pmatrix},$$
$$R_{ab}=\begin{pmatrix} 0 & 0 & 0\\ 0 & 0 & 0\\ 0 & 0 & 0 \end{pmatrix},\quad R=0.$$

    
    \hypertarget{spherical-coordinate}{%
\paragraph{Spherical coordinate}\label{spherical-coordinate}}

In \(3\)-dimensional flat space \(\mathbb{R}^3\), the line element in
spherical coordinate is \[
ds^2 = dr^2 + r^2d\theta^2 + r^2\sin^2\theta d\varphi^2
\] then we can calculate the \textbf{\emph{Christoffel symbols}},
\textbf{\emph{Riemann tensor}}, \textbf{\emph{Ricci tensor}} and
\textbf{\emph{Ricci scalar}}.

    \begin{tcolorbox}[breakable, size=fbox, boxrule=1pt, pad at break*=1mm,colback=cellbackground, colframe=cellborder]
\prompt{In}{incolor}{7}{\boxspacing}
\begin{Verbatim}[commandchars=\\\{\}]
\PY{n}{r}\PY{p}{,} \PY{n}{theta}\PY{p}{,} \PY{n}{phi} \PY{o}{=} \PY{n}{sp}\PY{o}{.}\PY{n}{symbols}\PY{p}{(}\PY{l+s+s1}{\PYZsq{}}\PY{l+s+s1}{r theta varphi}\PY{l+s+s1}{\PYZsq{}}\PY{p}{)}
\PY{n}{coords} \PY{o}{=} \PY{n}{sp}\PY{o}{.}\PY{n}{Matrix}\PY{p}{(}\PY{p}{[}\PY{n}{r}\PY{p}{,} \PY{n}{theta}\PY{p}{,} \PY{n}{phi}\PY{p}{]}\PY{p}{)}
\PY{n}{metric} \PY{o}{=} \PY{n}{sp}\PY{o}{.}\PY{n}{diag}\PY{p}{(}\PY{l+m+mi}{1}\PY{p}{,} \PY{n}{r}\PY{o}{*}\PY{o}{*}\PY{l+m+mi}{2}\PY{p}{,} \PY{n}{r}\PY{o}{*}\PY{o}{*}\PY{l+m+mi}{2} \PY{o}{*} \PY{p}{(}\PY{n}{sp}\PY{o}{.}\PY{n}{sin}\PY{p}{(}\PY{n}{theta}\PY{p}{)}\PY{p}{)}\PY{o}{*}\PY{o}{*}\PY{l+m+mi}{2}\PY{p}{)}
\PY{n}{geo} \PY{o}{=} \PY{n}{RiemannGeometry}\PY{p}{(}\PY{n}{metric}\PY{p}{,} \PY{n}{coords}\PY{p}{)}
\PY{n}{geo}\PY{o}{.}\PY{n}{calculate}\PY{p}{(}\PY{p}{)}
\PY{n}{geo}
\end{Verbatim}
\end{tcolorbox}

    \begin{Verbatim}[commandchars=\\\{\}]
Calculating {\ldots}
    \end{Verbatim}

    \begin{Verbatim}[commandchars=\\\{\}]
Christoffel Symbols : 100\%|\textblock\textblock\textblock\textblock\textblock\textblock\textblock\textblock\textblock\textblock\textblock\textblock\textblock\textblock\textblock\textblock\textblock\textblock\textblock\textblock\textblock\textblock\textblock\textblock\textblock\textblock\textblock\textblock\textblock\textblock\textblock\textblock\textblock\textblock\textblock| 3/3[00:00<00:00, 45.22it/s]
Riemann Tensor      : 100\%|\textblock\textblock\textblock\textblock\textblock\textblock\textblock\textblock\textblock\textblock\textblock\textblock\textblock\textblock\textblock\textblock\textblock\textblock\textblock\textblock\textblock\textblock\textblock\textblock\textblock\textblock\textblock\textblock\textblock\textblock\textblock\textblock\textblock\textblock\textblock| 3/3[00:00<00:00, 95.38it/s]
Ricci Tensor        : 100\%|\textblock\textblock\textblock\textblock\textblock\textblock\textblock\textblock\textblock\textblock\textblock\textblock\textblock\textblock\textblock\textblock\textblock\textblock\textblock\textblock\textblock\textblock\textblock\textblock\textblock\textblock\textblock\textblock\textblock\textblock\textblock\textblock\textblock| 3/3 [00:00<00:00,
5094.30it/s]
Ricci Scalar        : 100\%|\textblock\textblock\textblock\textblock\textblock\textblock\textblock\textblock\textblock\textblock\textblock\textblock\textblock\textblock\textblock\textblock\textblock\textblock\textblock\textblock\textblock\textblock\textblock\textblock\textblock\textblock\textblock\textblock\textblock\textblock\textblock\textblock| 3/3 [00:00<00:00,
45425.68it/s]
    \end{Verbatim}
 
            
\prompt{Out}{outcolor}{7}{}
    
    $$x^{a}=\begin{pmatrix}r\\\theta\\\varphi\end{pmatrix},\quad g_{ab}=\begin{pmatrix} 1 & 0 & 0\\ 0 & r^{2} & 0\\ 0 & 0 & r^{2} \sin^{2}{\left(\theta \right)} \end{pmatrix},\quad g^{ab}=\begin{pmatrix} 1 & 0 & 0\\ 0 & \frac{1}{r^{2}} & 0\\ 0 & 0 & \frac{1}{r^{2} \sin^{2}{\left(\theta \right)}} \end{pmatrix},$$
$${\Gamma^{a}}_{bc}=\begin{pmatrix} \begin{pmatrix} 0 & 0 & 0\\ 0 & - r & 0\\ 0 & 0 & - r \sin^{2}{\left(\theta \right)} \end{pmatrix} & \begin{pmatrix} 0 & \frac{1}{r} & 0\\ \frac{1}{r} & 0 & 0\\ 0 & 0 & - \sin{\left(\theta \right)} \cos{\left(\theta \right)} \end{pmatrix} & \begin{pmatrix} 0 & 0 & \frac{1}{r}\\ 0 & 0 & \frac{1}{\tan{\left(\theta \right)}}\\ \frac{1}{r} & \frac{1}{\tan{\left(\theta \right)}} & 0 \end{pmatrix} \end{pmatrix},$$
$${R^{a}}_{bcd}=\begin{pmatrix} \begin{pmatrix} 0 & 0 & 0\\ 0 & 0 & 0\\ 0 & 0 & 0 \end{pmatrix} & \begin{pmatrix} 0 & 0 & 0\\ 0 & 0 & 0\\ 0 & 0 & 0 \end{pmatrix} & \begin{pmatrix} 0 & 0 & 0\\ 0 & 0 & 0\\ 0 & 0 & 0 \end{pmatrix}\\ \begin{pmatrix} 0 & 0 & 0\\ 0 & 0 & 0\\ 0 & 0 & 0 \end{pmatrix} & \begin{pmatrix} 0 & 0 & 0\\ 0 & 0 & 0\\ 0 & 0 & 0 \end{pmatrix} & \begin{pmatrix} 0 & 0 & 0\\ 0 & 0 & 0\\ 0 & 0 & 0 \end{pmatrix}\\ \begin{pmatrix} 0 & 0 & 0\\ 0 & 0 & 0\\ 0 & 0 & 0 \end{pmatrix} & \begin{pmatrix} 0 & 0 & 0\\ 0 & 0 & 0\\ 0 & 0 & 0 \end{pmatrix} & \begin{pmatrix} 0 & 0 & 0\\ 0 & 0 & 0\\ 0 & 0 & 0 \end{pmatrix} \end{pmatrix},$$
$$R_{ab}=\begin{pmatrix} 0 & 0 & 0\\ 0 & 0 & 0\\ 0 & 0 & 0 \end{pmatrix},\quad R=0.$$

    \hypertarget{cylindrical-coordinate}{%
\paragraph{Cylindrical coordinate}\label{cylindrical-coordinate}}

In \(3\)-dimensional flat space \(\mathbb{R}^3\), the line element in
cylindrical coordinate is \[
ds^2 = dr^2 + r^2d\varphi^2 + dz^2
\] then we can calculate the \textbf{\emph{Christoffel symbols}},
\textbf{\emph{Riemann tensor}}, \textbf{\emph{Ricci tensor}} and
\textbf{\emph{Ricci scalar}}.

    \begin{tcolorbox}[breakable, size=fbox, boxrule=1pt, pad at break*=1mm,colback=cellbackground, colframe=cellborder]
\prompt{In}{incolor}{8}{\boxspacing}
\begin{Verbatim}[commandchars=\\\{\}]
\PY{n}{r}\PY{p}{,} \PY{n}{phi}\PY{p}{,} \PY{n}{z} \PY{o}{=} \PY{n}{sp}\PY{o}{.}\PY{n}{symbols}\PY{p}{(}\PY{l+s+s1}{\PYZsq{}}\PY{l+s+s1}{r theta z}\PY{l+s+s1}{\PYZsq{}}\PY{p}{)}
\PY{n}{coords} \PY{o}{=} \PY{n}{sp}\PY{o}{.}\PY{n}{Matrix}\PY{p}{(}\PY{p}{[}\PY{n}{r}\PY{p}{,} \PY{n}{phi}\PY{p}{,} \PY{n}{z}\PY{p}{]}\PY{p}{)}
\PY{n}{metric} \PY{o}{=} \PY{n}{sp}\PY{o}{.}\PY{n}{diag}\PY{p}{(}\PY{l+m+mi}{1}\PY{p}{,} \PY{n}{r}\PY{o}{*}\PY{o}{*}\PY{l+m+mi}{2}\PY{p}{,} \PY{l+m+mi}{1}\PY{p}{)}
\PY{n}{geo} \PY{o}{=} \PY{n}{RiemannGeometry}\PY{p}{(}\PY{n}{metric}\PY{p}{,} \PY{n}{coords}\PY{p}{)}
\PY{n}{geo}\PY{o}{.}\PY{n}{calculate}\PY{p}{(}\PY{p}{)}
\PY{n}{geo}
\end{Verbatim}
\end{tcolorbox}

    \begin{Verbatim}[commandchars=\\\{\}]
Calculating {\ldots}
    \end{Verbatim}

    \begin{Verbatim}[commandchars=\\\{\}]
Christoffel Symbols : 100\%|\textblock\textblock\textblock\textblock\textblock\textblock\textblock\textblock\textblock\textblock\textblock\textblock\textblock\textblock\textblock\textblock\textblock\textblock\textblock\textblock\textblock\textblock\textblock\textblock\textblock\textblock\textblock\textblock\textblock\textblock\textblock\textblock\textblock\textblock| 3/3 [00:00<00:00,369.50it/s]
Riemann Tensor      : 100\%|\textblock\textblock\textblock\textblock\textblock\textblock\textblock\textblock\textblock\textblock\textblock\textblock\textblock\textblock\textblock\textblock\textblock\textblock\textblock\textblock\textblock\textblock\textblock\textblock\textblock\textblock\textblock\textblock\textblock\textblock\textblock\textblock\textblock\textblock| 3/3 [00:00<00:00,360.67it/s]
Ricci Tensor        : 100\%|\textblock\textblock\textblock\textblock\textblock\textblock\textblock\textblock\textblock\textblock\textblock\textblock\textblock\textblock\textblock\textblock\textblock\textblock\textblock\textblock\textblock\textblock\textblock\textblock\textblock\textblock\textblock\textblock\textblock\textblock\textblock\textblock\textblock| 3/3 [00:00<00:00,3424.85it/s]
Ricci Scalar        : 100\%|\textblock\textblock\textblock\textblock\textblock\textblock\textblock\textblock\textblock\textblock\textblock\textblock\textblock\textblock\textblock\textblock\textblock\textblock\textblock\textblock\textblock\textblock\textblock\textblock\textblock\textblock\textblock\textblock\textblock\textblock\textblock\textblock| 3/3 [00:00<00:00,41120.63it/s]
    \end{Verbatim}
 
            
\prompt{Out}{outcolor}{8}{}
    
    $$x^{a}=\begin{pmatrix}r\\\theta\\z\end{pmatrix},\quad g_{ab}=\begin{pmatrix} 1 & 0 & 0\\ 0 & r^{2} & 0\\ 0 & 0 & 1 \end{pmatrix},\quad g^{ab}=\begin{pmatrix} 1 & 0 & 0\\ 0 & \frac{1}{r^{2}} & 0\\ 0 & 0 & 1 \end{pmatrix},$$
$${\Gamma^{a}}_{bc}=\begin{pmatrix} \begin{pmatrix} 0 & 0 & 0\\ 0 & - r & 0\\ 0 & 0 & 0 \end{pmatrix} & \begin{pmatrix} 0 & \frac{1}{r} & 0\\ \frac{1}{r} & 0 & 0\\ 0 & 0 & 0 \end{pmatrix} & \begin{pmatrix} 0 & 0 & 0\\ 0 & 0 & 0\\ 0 & 0 & 0 \end{pmatrix} \end{pmatrix},$$
$${R^{a}}_{bcd}=\begin{pmatrix} \begin{pmatrix} 0 & 0 & 0\\ 0 & 0 & 0\\ 0 & 0 & 0 \end{pmatrix} & \begin{pmatrix} 0 & 0 & 0\\ 0 & 0 & 0\\ 0 & 0 & 0 \end{pmatrix} & \begin{pmatrix} 0 & 0 & 0\\ 0 & 0 & 0\\ 0 & 0 & 0 \end{pmatrix}\\ \begin{pmatrix} 0 & 0 & 0\\ 0 & 0 & 0\\ 0 & 0 & 0 \end{pmatrix} & \begin{pmatrix} 0 & 0 & 0\\ 0 & 0 & 0\\ 0 & 0 & 0 \end{pmatrix} & \begin{pmatrix} 0 & 0 & 0\\ 0 & 0 & 0\\ 0 & 0 & 0 \end{pmatrix}\\ \begin{pmatrix} 0 & 0 & 0\\ 0 & 0 & 0\\ 0 & 0 & 0 \end{pmatrix} & \begin{pmatrix} 0 & 0 & 0\\ 0 & 0 & 0\\ 0 & 0 & 0 \end{pmatrix} & \begin{pmatrix} 0 & 0 & 0\\ 0 & 0 & 0\\ 0 & 0 & 0 \end{pmatrix} \end{pmatrix},$$
$$R_{ab}=\begin{pmatrix} 0 & 0 & 0\\ 0 & 0 & 0\\ 0 & 0 & 0 \end{pmatrix},\quad R=0.$$  

    \hypertarget{curved-manifolds}{%
\subsubsection{Curved Manifolds}\label{curved-manifolds}}

    \hypertarget{sphere-s2}{%
\paragraph{\texorpdfstring{Sphere
\(S^2\)}{Sphere S\^{}2}}\label{sphere-s2}}

On a sphere, the line element is given by \[
ds^2 = dr^2 + R^2 d\theta^2 + R^2 \sin^2\theta d\varphi,
\] then we can calculate the \textbf{\emph{Christoffel symbols}},
\textbf{\emph{Riemann tensor}}, \textbf{\emph{Ricci tensor}} and
\textbf{\emph{Ricci scalar}}.

    \begin{tcolorbox}[breakable, size=fbox, boxrule=1pt, pad at break*=1mm,colback=cellbackground, colframe=cellborder]
\prompt{In}{incolor}{9}{\boxspacing}
\begin{Verbatim}[commandchars=\\\{\}]
\PY{n}{theta}\PY{p}{,} \PY{n}{phi} \PY{o}{=} \PY{n}{sp}\PY{o}{.}\PY{n}{symbols}\PY{p}{(}\PY{l+s+s1}{\PYZsq{}}\PY{l+s+s1}{theta varphi}\PY{l+s+s1}{\PYZsq{}}\PY{p}{)}
\PY{n}{R} \PY{o}{=} \PY{n}{sp}\PY{o}{.}\PY{n}{symbols}\PY{p}{(}\PY{l+s+s1}{\PYZsq{}}\PY{l+s+s1}{R}\PY{l+s+s1}{\PYZsq{}}\PY{p}{)}
\PY{n}{coords} \PY{o}{=} \PY{n}{sp}\PY{o}{.}\PY{n}{Matrix}\PY{p}{(}\PY{p}{[}\PY{n}{theta}\PY{p}{,} \PY{n}{phi}\PY{p}{]}\PY{p}{)}
\PY{n}{metric} \PY{o}{=} \PY{n}{sp}\PY{o}{.}\PY{n}{diag}\PY{p}{(}\PY{n}{R}\PY{o}{*}\PY{o}{*}\PY{l+m+mi}{2}\PY{p}{,} \PY{n}{R}\PY{o}{*}\PY{o}{*}\PY{l+m+mi}{2} \PY{o}{*} \PY{p}{(}\PY{n}{sp}\PY{o}{.}\PY{n}{sin}\PY{p}{(}\PY{n}{theta}\PY{p}{)}\PY{p}{)}\PY{o}{*}\PY{o}{*}\PY{l+m+mi}{2}\PY{p}{)}
\PY{n}{geo} \PY{o}{=} \PY{n}{RiemannGeometry}\PY{p}{(}\PY{n}{metric}\PY{p}{,} \PY{n}{coords}\PY{p}{)}
\PY{n}{geo}\PY{o}{.}\PY{n}{calculate}\PY{p}{(}\PY{p}{)}
\PY{n}{geo}
\end{Verbatim}
\end{tcolorbox}

    \begin{Verbatim}[commandchars=\\\{\}]
Calculating {\ldots}
    \end{Verbatim}

    \begin{Verbatim}[commandchars=\\\{\}]
Christoffel Symbols : 100\%|\textblock\textblock\textblock\textblock\textblock\textblock\textblock\textblock\textblock\textblock\textblock\textblock\textblock\textblock\textblock\textblock\textblock\textblock\textblock\textblock\textblock\textblock\textblock\textblock\textblock\textblock\textblock\textblock\textblock\textblock\textblock\textblock\textblock\textblock| 2/2 [00:00<00:00,105.26it/s]
Riemann Tensor      : 100\%|\textblock\textblock\textblock\textblock\textblock\textblock\textblock\textblock\textblock\textblock\textblock\textblock\textblock\textblock\textblock\textblock\textblock\textblock\textblock\textblock\textblock\textblock\textblock\textblock\textblock\textblock\textblock\textblock\textblock\textblock\textblock\textblock\textblock\textblock\textblock| 2/2[00:00<00:00, 62.22it/s]
Ricci Tensor        : 100\%|\textblock\textblock\textblock\textblock\textblock\textblock\textblock\textblock\textblock\textblock\textblock\textblock\textblock\textblock\textblock\textblock\textblock\textblock\textblock\textblock\textblock\textblock\textblock\textblock\textblock\textblock\textblock\textblock\textblock\textblock\textblock\textblock\textblock\textblock| 2/2 [00:00<00:00,336.35it/s]
Ricci Scalar        : 100\%|\textblock\textblock\textblock\textblock\textblock\textblock\textblock\textblock\textblock\textblock\textblock\textblock\textblock\textblock\textblock\textblock\textblock\textblock\textblock\textblock\textblock\textblock\textblock\textblock\textblock\textblock\textblock\textblock\textblock\textblock\textblock\textblock| 2/2 [00:00<00:00,13421.77it/s]
    \end{Verbatim}
 
            
\prompt{Out}{outcolor}{9}{}
    
    $$x^{a}=\begin{pmatrix}\theta\\\varphi\end{pmatrix},\quad g_{ab}=\begin{pmatrix} R^{2} & 0\\ 0 & R^{2} \sin^{2}{\left(\theta \right)} \end{pmatrix},\quad g^{ab}=\begin{pmatrix} \frac{1}{R^{2}} & 0\\ 0 & \frac{1}{R^{2} \sin^{2}{\left(\theta \right)}} \end{pmatrix},$$
$${\Gamma^{a}}_{bc}=\begin{pmatrix} \begin{pmatrix} 0 & 0\\ 0 & - \sin{\left(\theta \right)} \cos{\left(\theta \right)} \end{pmatrix} & \begin{pmatrix} 0 & \frac{1}{\tan{\left(\theta \right)}}\\ \frac{1}{\tan{\left(\theta \right)}} & 0 \end{pmatrix} \end{pmatrix},$$
$${R^{a}}_{bcd}=\begin{pmatrix} \begin{pmatrix} 0 & 0\\ 0 & 0 \end{pmatrix} & \begin{pmatrix} 0 & \sin^{2}{\left(\theta \right)}\\ - \sin^{2}{\left(\theta \right)} & 0 \end{pmatrix}\\ \begin{pmatrix} 0 & -1\\ 1 & 0 \end{pmatrix} & \begin{pmatrix} 0 & 0\\ 0 & 0 \end{pmatrix} \end{pmatrix},$$
$$R_{ab}=\begin{pmatrix} 1 & 0\\ 0 & \sin^{2}{\left(\theta \right)} \end{pmatrix},\quad R=\frac{2}{R^{2}}.$$

\newpage %%%%%%%%%%%%%%%%%%%%%%%%%%%%%%%%%%%%%%%%%%%%%%%%%%%%%%%%

    \hypertarget{torus-t2}{%
\paragraph{\texorpdfstring{Torus
\(T^2\)}{Torus T\^{}2}}\label{torus-t2}}

On a sphere, the line element is given by \[
ds^2 = \left(R + r\cos v\right)^2 du^2 + r^2 dv^2,
\] then we can calculate the \textbf{\emph{Christoffel symbols}},
\textbf{\emph{Riemann tensor}}, \textbf{\emph{Ricci tensor}} and
\textbf{\emph{Ricci scalar}}.

    \begin{tcolorbox}[breakable, size=fbox, boxrule=1pt, pad at break*=1mm,colback=cellbackground, colframe=cellborder]
\prompt{In}{incolor}{10}{\boxspacing}
\begin{Verbatim}[commandchars=\\\{\}]
\PY{n}{u}\PY{p}{,} \PY{n}{v} \PY{o}{=} \PY{n}{sp}\PY{o}{.}\PY{n}{symbols}\PY{p}{(}\PY{l+s+s1}{\PYZsq{}}\PY{l+s+s1}{u v}\PY{l+s+s1}{\PYZsq{}}\PY{p}{)}
\PY{n}{R}\PY{p}{,} \PY{n}{r} \PY{o}{=} \PY{n}{sp}\PY{o}{.}\PY{n}{symbols}\PY{p}{(}\PY{l+s+s1}{\PYZsq{}}\PY{l+s+s1}{R, r}\PY{l+s+s1}{\PYZsq{}}\PY{p}{)}
\PY{n}{coords} \PY{o}{=} \PY{n}{sp}\PY{o}{.}\PY{n}{Matrix}\PY{p}{(}\PY{p}{[}\PY{n}{u}\PY{p}{,} \PY{n}{v}\PY{p}{]}\PY{p}{)}
\PY{n}{metric} \PY{o}{=} \PY{n}{sp}\PY{o}{.}\PY{n}{diag}\PY{p}{(}\PY{p}{(}\PY{n}{R} \PY{o}{+} \PY{n}{r}\PY{o}{*}\PY{n}{sp}\PY{o}{.}\PY{n}{cos}\PY{p}{(}\PY{n}{v}\PY{p}{)}\PY{p}{)}\PY{o}{*}\PY{o}{*}\PY{l+m+mi}{2}\PY{p}{,} \PY{n}{r}\PY{o}{*}\PY{o}{*}\PY{l+m+mi}{2}\PY{p}{)}
\PY{n}{geo} \PY{o}{=} \PY{n}{RiemannGeometry}\PY{p}{(}\PY{n}{metric}\PY{p}{,} \PY{n}{coords}\PY{p}{)}
\PY{n}{geo}\PY{o}{.}\PY{n}{calculate}\PY{p}{(}\PY{p}{)}
\PY{n}{geo}
\end{Verbatim}
\end{tcolorbox}

    \begin{Verbatim}[commandchars=\\\{\}]
Calculating {\ldots}
    \end{Verbatim}

    \begin{Verbatim}[commandchars=\\\{\}]
Christoffel Symbols : 100\%|\textblock\textblock\textblock\textblock\textblock\textblock\textblock\textblock\textblock\textblock\textblock\textblock\textblock\textblock\textblock\textblock\textblock\textblock\textblock\textblock\textblock\textblock\textblock\textblock\textblock\textblock\textblock\textblock\textblock\textblock\textblock\textblock\textblock\textblock\textblock| 2/2[00:00<00:00, 12.76it/s]
Riemann Tensor      : 100\%|\textblock\textblock\textblock\textblock\textblock\textblock\textblock\textblock\textblock\textblock\textblock\textblock\textblock\textblock\textblock\textblock\textblock\textblock\textblock\textblock\textblock\textblock\textblock\textblock\textblock\textblock\textblock\textblock\textblock\textblock\textblock\textblock\textblock\textblock\textblock| 2/2[00:00<00:00, 13.72it/s]
Ricci Tensor        : 100\%|\textblock\textblock\textblock\textblock\textblock\textblock\textblock\textblock\textblock\textblock\textblock\textblock\textblock\textblock\textblock\textblock\textblock\textblock\textblock\textblock\textblock\textblock\textblock\textblock\textblock\textblock\textblock\textblock\textblock\textblock\textblock\textblock\textblock\textblock\textblock| 2/2[00:00<00:00, 33.62it/s]
Ricci Scalar        : 100\%|\textblock\textblock\textblock\textblock\textblock\textblock\textblock\textblock\textblock\textblock\textblock\textblock\textblock\textblock\textblock\textblock\textblock\textblock\textblock\textblock\textblock\textblock\textblock\textblock\textblock\textblock\textblock\textblock\textblock\textblock\textblock\textblock\textblock| 2/2 [00:00<00:00,
8971.77it/s]
    \end{Verbatim}
 
            
\prompt{Out}{outcolor}{10}{}
    
    $$x^{a}=\begin{pmatrix}u\\v\end{pmatrix},\quad g_{ab}=\begin{pmatrix} \left(R + r \cos{\left(v \right)}\right)^{2} & 0\\ 0 & r^{2} \end{pmatrix},\quad g^{ab}=\begin{pmatrix} \frac{1}{R^{2} + 2 R r \cos{\left(v \right)} + r^{2} \cos^{2}{\left(v \right)}} & 0\\ 0 & \frac{1}{r^{2}} \end{pmatrix},$$
$${\Gamma^{a}}_{bc}=\begin{pmatrix} \begin{pmatrix} 0 & - \frac{r \sin{\left(v \right)}}{R + r \cos{\left(v \right)}}\\ - \frac{r \sin{\left(v \right)}}{R + r \cos{\left(v \right)}} & 0 \end{pmatrix} & \begin{pmatrix} \frac{\left(R + r \cos{\left(v \right)}\right) \sin{\left(v \right)}}{r} & 0\\ 0 & 0 \end{pmatrix} \end{pmatrix},$$
$${R^{a}}_{bcd}=\begin{pmatrix} \begin{pmatrix} 0 & 0\\ 0 & 0 \end{pmatrix} & \begin{pmatrix} 0 & \frac{r \cos{\left(v \right)}}{R + r \cos{\left(v \right)}}\\ - \frac{r \cos{\left(v \right)}}{R + r \cos{\left(v \right)}} & 0 \end{pmatrix}\\ \begin{pmatrix} 0 & - \frac{\left(R + r \cos{\left(v \right)}\right) \cos{\left(v \right)}}{r}\\ \frac{\left(R + r \cos{\left(v \right)}\right) \cos{\left(v \right)}}{r} & 0 \end{pmatrix} & \begin{pmatrix} 0 & 0\\ 0 & 0 \end{pmatrix} \end{pmatrix},$$
$$R_{ab}=\begin{pmatrix} \frac{\left(R + r \cos{\left(v \right)}\right) \cos{\left(v \right)}}{r} & 0\\ 0 & \frac{r \cos{\left(v \right)}}{R + r \cos{\left(v \right)}} \end{pmatrix},\quad R=\frac{2 \cos{\left(v \right)}}{r \left(R + r \cos{\left(v \right)}\right)}.$$

    \hypertarget{example-for-manifolds-in-spacetime-mathbbrn1}{%
\subsection{\texorpdfstring{Example for Manifolds in Spacetime
\(\mathbb{R}^{n+1}\)}{Example for Manifolds in Spacetime \textbackslash mathbb\{R\}\^{}\{n+1\}}}\label{example-for-manifolds-in-spacetime-mathbbrn1}}

    \hypertarget{flat-manifolds-in-mathbbr31}{%
\subsubsection{\texorpdfstring{Flat Manifolds in
\(\mathbb{R}^{3+1}\)}{Flat Manifolds in \textbackslash mathbb\{R\}\^{}\{3+1\}}}\label{flat-manifolds-in-mathbbr31}}

    \hypertarget{cartesian-coordinate}{%
\paragraph{Cartesian coordinate}\label{cartesian-coordinate}}

Minkowski metric is the metric in flat spacetime, the line element in
Cartesian coordinate is given by \[
ds^2 = -dt^2 + dx^2 + dy^2 + dz^2
\] then we can calculate the \textbf{\emph{Christoffel symbols}},
\textbf{\emph{Riemann tensor}}, \textbf{\emph{Ricci tensor}} and
\textbf{\emph{Ricci scalar}}.

\begin{quote}
Here using Geometrized unit system \(G=c=1\).
\end{quote}

    \begin{tcolorbox}[breakable, size=fbox, boxrule=1pt, pad at break*=1mm,colback=cellbackground, colframe=cellborder]
\prompt{In}{incolor}{15}{\boxspacing}
\begin{Verbatim}[commandchars=\\\{\}]
\PY{n}{t}\PY{p}{,} \PY{n}{x}\PY{p}{,} \PY{n}{y}\PY{p}{,} \PY{n}{z} \PY{o}{=} \PY{n}{sp}\PY{o}{.}\PY{n}{symbols}\PY{p}{(}\PY{l+s+s1}{\PYZsq{}}\PY{l+s+s1}{t x y z}\PY{l+s+s1}{\PYZsq{}}\PY{p}{)}
\PY{n}{coords} \PY{o}{=} \PY{n}{sp}\PY{o}{.}\PY{n}{Matrix}\PY{p}{(}\PY{p}{[}\PY{n}{t}\PY{p}{,} \PY{n}{x}\PY{p}{,} \PY{n}{y}\PY{p}{,} \PY{n}{z}\PY{p}{]}\PY{p}{)}
\PY{n}{metric} \PY{o}{=} \PY{n}{sp}\PY{o}{.}\PY{n}{diag}\PY{p}{(}\PY{o}{\PYZhy{}}\PY{l+m+mi}{1}\PY{p}{,} \PY{l+m+mi}{1}\PY{p}{,} \PY{l+m+mi}{1}\PY{p}{,} \PY{l+m+mi}{1}\PY{p}{)}
\PY{n}{geo} \PY{o}{=} \PY{n}{RiemannGeometry}\PY{p}{(}\PY{n}{metric}\PY{p}{,} \PY{n}{coords}\PY{p}{)}
\PY{n}{geo}\PY{o}{.}\PY{n}{calculate}\PY{p}{(}\PY{p}{)}
\PY{n}{geo}
\end{Verbatim}
\end{tcolorbox}

    \begin{Verbatim}[commandchars=\\\{\}]
Calculating {\ldots}
    \end{Verbatim}

    \begin{Verbatim}[commandchars=\\\{\}]
Christoffel Symbols : 100\%|\textblock\textblock\textblock\textblock\textblock\textblock\textblock\textblock\textblock\textblock\textblock\textblock\textblock\textblock\textblock\textblock\textblock\textblock\textblock\textblock\textblock\textblock\textblock\textblock\textblock\textblock\textblock\textblock\textblock\textblock\textblock\textblock\textblock\textblock| 4/4 [00:00<00:00,234.91it/s]
Riemann Tensor      : 100\%|\textblock\textblock\textblock\textblock\textblock\textblock\textblock\textblock\textblock\textblock\textblock\textblock\textblock\textblock\textblock\textblock\textblock\textblock\textblock\textblock\textblock\textblock\textblock\textblock\textblock\textblock\textblock\textblock\textblock\textblock\textblock\textblock\textblock\textblock| 4/4 [00:00<00:00,132.96it/s]
Ricci Tensor        : 100\%|\textblock\textblock\textblock\textblock\textblock\textblock\textblock\textblock\textblock\textblock\textblock\textblock\textblock\textblock\textblock\textblock\textblock\textblock\textblock\textblock\textblock\textblock\textblock\textblock\textblock\textblock\textblock\textblock\textblock\textblock\textblock\textblock\textblock| 4/4 [00:00<00:00,2240.85it/s]
Ricci Scalar        : 100\%|\textblock\textblock\textblock\textblock\textblock\textblock\textblock\textblock\textblock\textblock\textblock\textblock\textblock\textblock\textblock\textblock\textblock\textblock\textblock\textblock\textblock\textblock\textblock\textblock\textblock\textblock\textblock\textblock\textblock\textblock\textblock\textblock| 4/4 [00:00<00:00,45964.98it/s]
    \end{Verbatim}
 
            
\prompt{Out}{outcolor}{15}{}
    
    $$x^{a}=\begin{pmatrix}t\\x\\y\\z\end{pmatrix},\quad g_{ab}=\begin{pmatrix} -1 & 0 & 0 & 0\\ 0 & 1 & 0 & 0\\ 0 & 0 & 1 & 0\\ 0 & 0 & 0 & 1 \end{pmatrix},\quad g^{ab}=\begin{pmatrix} -1 & 0 & 0 & 0\\ 0 & 1 & 0 & 0\\ 0 & 0 & 1 & 0\\ 0 & 0 & 0 & 1 \end{pmatrix},$$
$${\Gamma^{a}}_{bc}=\begin{pmatrix} \begin{pmatrix} 0 & 0 & 0 & 0\\ 0 & 0 & 0 & 0\\ 0 & 0 & 0 & 0\\ 0 & 0 & 0 & 0 \end{pmatrix} & \begin{pmatrix} 0 & 0 & 0 & 0\\ 0 & 0 & 0 & 0\\ 0 & 0 & 0 & 0\\ 0 & 0 & 0 & 0 \end{pmatrix} & \begin{pmatrix} 0 & 0 & 0 & 0\\ 0 & 0 & 0 & 0\\ 0 & 0 & 0 & 0\\ 0 & 0 & 0 & 0 \end{pmatrix} & \begin{pmatrix} 0 & 0 & 0 & 0\\ 0 & 0 & 0 & 0\\ 0 & 0 & 0 & 0\\ 0 & 0 & 0 & 0 \end{pmatrix} \end{pmatrix},$$
$${R^{a}}_{bcd}=\begin{pmatrix} \begin{pmatrix} 0 & 0 & 0 & 0\\ 0 & 0 & 0 & 0\\ 0 & 0 & 0 & 0\\ 0 & 0 & 0 & 0 \end{pmatrix} & \begin{pmatrix} 0 & 0 & 0 & 0\\ 0 & 0 & 0 & 0\\ 0 & 0 & 0 & 0\\ 0 & 0 & 0 & 0 \end{pmatrix} & \begin{pmatrix} 0 & 0 & 0 & 0\\ 0 & 0 & 0 & 0\\ 0 & 0 & 0 & 0\\ 0 & 0 & 0 & 0 \end{pmatrix} & \begin{pmatrix} 0 & 0 & 0 & 0\\ 0 & 0 & 0 & 0\\ 0 & 0 & 0 & 0\\ 0 & 0 & 0 & 0 \end{pmatrix}\\ \begin{pmatrix} 0 & 0 & 0 & 0\\ 0 & 0 & 0 & 0\\ 0 & 0 & 0 & 0\\ 0 & 0 & 0 & 0 \end{pmatrix} & \begin{pmatrix} 0 & 0 & 0 & 0\\ 0 & 0 & 0 & 0\\ 0 & 0 & 0 & 0\\ 0 & 0 & 0 & 0 \end{pmatrix} & \begin{pmatrix} 0 & 0 & 0 & 0\\ 0 & 0 & 0 & 0\\ 0 & 0 & 0 & 0\\ 0 & 0 & 0 & 0 \end{pmatrix} & \begin{pmatrix} 0 & 0 & 0 & 0\\ 0 & 0 & 0 & 0\\ 0 & 0 & 0 & 0\\ 0 & 0 & 0 & 0 \end{pmatrix}\\ \begin{pmatrix} 0 & 0 & 0 & 0\\ 0 & 0 & 0 & 0\\ 0 & 0 & 0 & 0\\ 0 & 0 & 0 & 0 \end{pmatrix} & \begin{pmatrix} 0 & 0 & 0 & 0\\ 0 & 0 & 0 & 0\\ 0 & 0 & 0 & 0\\ 0 & 0 & 0 & 0 \end{pmatrix} & \begin{pmatrix} 0 & 0 & 0 & 0\\ 0 & 0 & 0 & 0\\ 0 & 0 & 0 & 0\\ 0 & 0 & 0 & 0 \end{pmatrix} & \begin{pmatrix} 0 & 0 & 0 & 0\\ 0 & 0 & 0 & 0\\ 0 & 0 & 0 & 0\\ 0 & 0 & 0 & 0 \end{pmatrix}\\ \begin{pmatrix} 0 & 0 & 0 & 0\\ 0 & 0 & 0 & 0\\ 0 & 0 & 0 & 0\\ 0 & 0 & 0 & 0 \end{pmatrix} & \begin{pmatrix} 0 & 0 & 0 & 0\\ 0 & 0 & 0 & 0\\ 0 & 0 & 0 & 0\\ 0 & 0 & 0 & 0 \end{pmatrix} & \begin{pmatrix} 0 & 0 & 0 & 0\\ 0 & 0 & 0 & 0\\ 0 & 0 & 0 & 0\\ 0 & 0 & 0 & 0 \end{pmatrix} & \begin{pmatrix} 0 & 0 & 0 & 0\\ 0 & 0 & 0 & 0\\ 0 & 0 & 0 & 0\\ 0 & 0 & 0 & 0 \end{pmatrix} \end{pmatrix},$$
$$R_{ab}=\begin{pmatrix} 0 & 0 & 0 & 0\\ 0 & 0 & 0 & 0\\ 0 & 0 & 0 & 0\\ 0 & 0 & 0 & 0 \end{pmatrix},\quad R=0.$$

    \hypertarget{spheical-coordinate}{%
\paragraph{Spheical coordinate}\label{spheical-coordinate}}

Minkowski metric is the metric in flat spacetime, the line element in
spherical coordinate is given by \[
ds^2 = -dt^2 + dr^2 + r^2 d\theta^2 + r^2 \sin^2\theta d\varphi
\] then we can calculate the \textbf{\emph{Christoffel symbols}},
\textbf{\emph{Riemann tensor}}, \textbf{\emph{Ricci tensor}} and
\textbf{\emph{Ricci scalar}}.

\begin{quote}
Here using Geometrized unit system \(G=c=1\).
\end{quote}

    \begin{tcolorbox}[breakable, size=fbox, boxrule=1pt, pad at break*=1mm,colback=cellbackground, colframe=cellborder]
\prompt{In}{incolor}{16}{\boxspacing}
\begin{Verbatim}[commandchars=\\\{\}]
\PY{n}{t}\PY{p}{,} \PY{n}{r}\PY{p}{,} \PY{n}{theta}\PY{p}{,} \PY{n}{phi} \PY{o}{=} \PY{n}{sp}\PY{o}{.}\PY{n}{symbols}\PY{p}{(}\PY{l+s+s1}{\PYZsq{}}\PY{l+s+s1}{t r theta varphi}\PY{l+s+s1}{\PYZsq{}}\PY{p}{)}
\PY{n}{coords} \PY{o}{=} \PY{n}{sp}\PY{o}{.}\PY{n}{Matrix}\PY{p}{(}\PY{p}{[}\PY{n}{t}\PY{p}{,} \PY{n}{r}\PY{p}{,} \PY{n}{theta}\PY{p}{,} \PY{n}{phi}\PY{p}{]}\PY{p}{)}
\PY{n}{metric} \PY{o}{=} \PY{n}{sp}\PY{o}{.}\PY{n}{diag}\PY{p}{(}\PY{l+m+mi}{1}\PY{p}{,} \PY{o}{\PYZhy{}}\PY{l+m+mi}{1}\PY{p}{,} \PY{o}{\PYZhy{}}\PY{n}{r}\PY{o}{*}\PY{o}{*}\PY{l+m+mi}{2}\PY{p}{,} \PY{o}{\PYZhy{}}\PY{n}{r}\PY{o}{*}\PY{o}{*}\PY{l+m+mi}{2}\PY{o}{*}\PY{p}{(}\PY{n}{sp}\PY{o}{.}\PY{n}{sin}\PY{p}{(}\PY{n}{theta}\PY{p}{)}\PY{p}{)}\PY{o}{*}\PY{o}{*}\PY{l+m+mi}{2}\PY{p}{)}
\PY{n}{geo} \PY{o}{=} \PY{n}{RiemannGeometry}\PY{p}{(}\PY{n}{metric}\PY{p}{,} \PY{n}{coords}\PY{p}{)}
\PY{n}{geo}\PY{o}{.}\PY{n}{calculate}\PY{p}{(}\PY{p}{)}
\PY{n}{geo}
\end{Verbatim}
\end{tcolorbox}

    \begin{Verbatim}[commandchars=\\\{\}]
Calculating {\ldots}
    \end{Verbatim}

    \begin{Verbatim}[commandchars=\\\{\}]
Christoffel Symbols : 100\%|\textblock\textblock\textblock\textblock\textblock\textblock\textblock\textblock\textblock\textblock\textblock\textblock\textblock\textblock\textblock\textblock\textblock\textblock\textblock\textblock\textblock\textblock\textblock\textblock\textblock\textblock\textblock\textblock\textblock\textblock\textblock\textblock\textblock\textblock\textblock| 4/4[00:00<00:00, 65.89it/s]
Riemann Tensor      : 100\%|\textblock\textblock\textblock\textblock\textblock\textblock\textblock\textblock\textblock\textblock\textblock\textblock\textblock\textblock\textblock\textblock\textblock\textblock\textblock\textblock\textblock\textblock\textblock\textblock\textblock\textblock\textblock\textblock\textblock\textblock\textblock\textblock\textblock\textblock\textblock| 4/4[00:00<00:00, 67.29it/s]
Ricci Tensor        : 100\%|\textblock\textblock\textblock\textblock\textblock\textblock\textblock\textblock\textblock\textblock\textblock\textblock\textblock\textblock\textblock\textblock\textblock\textblock\textblock\textblock\textblock\textblock\textblock\textblock\textblock\textblock\textblock\textblock\textblock\textblock\textblock\textblock\textblock| 4/4 [00:00<00:00,3629.86it/s]
Ricci Scalar        : 100\%|\textblock\textblock\textblock\textblock\textblock\textblock\textblock\textblock\textblock\textblock\textblock\textblock\textblock\textblock\textblock\textblock\textblock\textblock\textblock\textblock\textblock\textblock\textblock\textblock\textblock\textblock\textblock\textblock\textblock\textblock\textblock\textblock| 4/4 [00:00<00:00,39475.80it/s]
    \end{Verbatim}
 
            
\prompt{Out}{outcolor}{16}{}
    
    $$x^{a}=\begin{pmatrix}t\\r\\\theta\\\varphi\end{pmatrix},\quad g_{ab}=\begin{pmatrix} 1 & 0 & 0 & 0\\ 0 & -1 & 0 & 0\\ 0 & 0 & - r^{2} & 0\\ 0 & 0 & 0 & - r^{2} \sin^{2}{\left(\theta \right)} \end{pmatrix},\quad g^{ab}=\begin{pmatrix} 1 & 0 & 0 & 0\\ 0 & -1 & 0 & 0\\ 0 & 0 & - \frac{1}{r^{2}} & 0\\ 0 & 0 & 0 & - \frac{1}{r^{2} \sin^{2}{\left(\theta \right)}} \end{pmatrix},$$
$${\Gamma^{a}}_{bc}=\begin{pmatrix} \begin{pmatrix} 0 & 0 & 0 & 0\\ 0 & 0 & 0 & 0\\ 0 & 0 & 0 & 0\\ 0 & 0 & 0 & 0 \end{pmatrix} & \begin{pmatrix} 0 & 0 & 0 & 0\\ 0 & 0 & 0 & 0\\ 0 & 0 & - r & 0\\ 0 & 0 & 0 & - r \sin^{2}{\left(\theta \right)} \end{pmatrix} & \begin{pmatrix} 0 & 0 & 0 & 0\\ 0 & 0 & \frac{1}{r} & 0\\ 0 & \frac{1}{r} & 0 & 0\\ 0 & 0 & 0 & - \sin{\left(\theta \right)} \cos{\left(\theta \right)} \end{pmatrix} & \begin{pmatrix} 0 & 0 & 0 & 0\\ 0 & 0 & 0 & \frac{1}{r}\\ 0 & 0 & 0 & \frac{1}{\tan{\left(\theta \right)}}\\ 0 & \frac{1}{r} & \frac{1}{\tan{\left(\theta \right)}} & 0 \end{pmatrix} \end{pmatrix},$$
$${R^{a}}_{bcd}=\begin{pmatrix} \begin{pmatrix} 0 & 0 & 0 & 0\\ 0 & 0 & 0 & 0\\ 0 & 0 & 0 & 0\\ 0 & 0 & 0 & 0 \end{pmatrix} & \begin{pmatrix} 0 & 0 & 0 & 0\\ 0 & 0 & 0 & 0\\ 0 & 0 & 0 & 0\\ 0 & 0 & 0 & 0 \end{pmatrix} & \begin{pmatrix} 0 & 0 & 0 & 0\\ 0 & 0 & 0 & 0\\ 0 & 0 & 0 & 0\\ 0 & 0 & 0 & 0 \end{pmatrix} & \begin{pmatrix} 0 & 0 & 0 & 0\\ 0 & 0 & 0 & 0\\ 0 & 0 & 0 & 0\\ 0 & 0 & 0 & 0 \end{pmatrix}\\ \begin{pmatrix} 0 & 0 & 0 & 0\\ 0 & 0 & 0 & 0\\ 0 & 0 & 0 & 0\\ 0 & 0 & 0 & 0 \end{pmatrix} & \begin{pmatrix} 0 & 0 & 0 & 0\\ 0 & 0 & 0 & 0\\ 0 & 0 & 0 & 0\\ 0 & 0 & 0 & 0 \end{pmatrix} & \begin{pmatrix} 0 & 0 & 0 & 0\\ 0 & 0 & 0 & 0\\ 0 & 0 & 0 & 0\\ 0 & 0 & 0 & 0 \end{pmatrix} & \begin{pmatrix} 0 & 0 & 0 & 0\\ 0 & 0 & 0 & 0\\ 0 & 0 & 0 & 0\\ 0 & 0 & 0 & 0 \end{pmatrix}\\ \begin{pmatrix} 0 & 0 & 0 & 0\\ 0 & 0 & 0 & 0\\ 0 & 0 & 0 & 0\\ 0 & 0 & 0 & 0 \end{pmatrix} & \begin{pmatrix} 0 & 0 & 0 & 0\\ 0 & 0 & 0 & 0\\ 0 & 0 & 0 & 0\\ 0 & 0 & 0 & 0 \end{pmatrix} & \begin{pmatrix} 0 & 0 & 0 & 0\\ 0 & 0 & 0 & 0\\ 0 & 0 & 0 & 0\\ 0 & 0 & 0 & 0 \end{pmatrix} & \begin{pmatrix} 0 & 0 & 0 & 0\\ 0 & 0 & 0 & 0\\ 0 & 0 & 0 & 0\\ 0 & 0 & 0 & 0 \end{pmatrix}\\ \begin{pmatrix} 0 & 0 & 0 & 0\\ 0 & 0 & 0 & 0\\ 0 & 0 & 0 & 0\\ 0 & 0 & 0 & 0 \end{pmatrix} & \begin{pmatrix} 0 & 0 & 0 & 0\\ 0 & 0 & 0 & 0\\ 0 & 0 & 0 & 0\\ 0 & 0 & 0 & 0 \end{pmatrix} & \begin{pmatrix} 0 & 0 & 0 & 0\\ 0 & 0 & 0 & 0\\ 0 & 0 & 0 & 0\\ 0 & 0 & 0 & 0 \end{pmatrix} & \begin{pmatrix} 0 & 0 & 0 & 0\\ 0 & 0 & 0 & 0\\ 0 & 0 & 0 & 0\\ 0 & 0 & 0 & 0 \end{pmatrix} \end{pmatrix},$$
$$R_{ab}=\begin{pmatrix} 0 & 0 & 0 & 0\\ 0 & 0 & 0 & 0\\ 0 & 0 & 0 & 0\\ 0 & 0 & 0 & 0 \end{pmatrix},\quad R=0.$$

\newpage %%%%%%%%%%%%%%%%%%%%%%%%%%%%%%%%%%%%%%%%%%%%%%%%%%%%%%%%   

    \hypertarget{curved-manifolds-in-mathbbr31}{%
\subsubsection{\texorpdfstring{Curved manifolds in
\(\mathbb{R}^{3+1}\)}{Curved manifolds in \textbackslash mathbb\{R\}\^{}\{3+1\}}}\label{curved-manifolds-in-mathbbr31}}

    \hypertarget{schwarzschild}{%
\paragraph{Schwarzschild}\label{schwarzschild}}

One of the solution for Einstein field equations, is Schwarzschild
metric, where the line element is given by \[
ds^2 = \left(1 - \frac{r_s}{r}\right) dt^2 - \left(1 - \frac{r_s}{r}\right)^{-1} dr^2 - r^2 d\theta^2 - r^2 \sin^2\theta d\varphi
\] then we can calculate the \textbf{\emph{Christoffel symbols}},
\textbf{\emph{Riemann tensor}}, \textbf{\emph{Ricci tensor}} and
\textbf{\emph{Ricci scalar}}.

\begin{quote}
Here using Geometrized unit system \(G=c=1\).
\end{quote}

    \begin{tcolorbox}[breakable, size=fbox, boxrule=1pt, pad at break*=1mm,colback=cellbackground, colframe=cellborder]
\prompt{In}{incolor}{11}{\boxspacing}
\begin{Verbatim}[commandchars=\\\{\}]
\PY{n}{t}\PY{p}{,} \PY{n}{r}\PY{p}{,} \PY{n}{theta}\PY{p}{,} \PY{n}{phi} \PY{o}{=} \PY{n}{sp}\PY{o}{.}\PY{n}{symbols}\PY{p}{(}\PY{l+s+s1}{\PYZsq{}}\PY{l+s+s1}{t r theta varphi}\PY{l+s+s1}{\PYZsq{}}\PY{p}{)}
\PY{n}{rs} \PY{o}{=} \PY{n}{sp}\PY{o}{.}\PY{n}{symbols}\PY{p}{(}\PY{l+s+s1}{\PYZsq{}}\PY{l+s+s1}{r\PYZus{}s}\PY{l+s+s1}{\PYZsq{}}\PY{p}{)}
\PY{n}{coords} \PY{o}{=} \PY{n}{sp}\PY{o}{.}\PY{n}{Matrix}\PY{p}{(}\PY{p}{[}\PY{n}{t}\PY{p}{,} \PY{n}{r}\PY{p}{,} \PY{n}{theta}\PY{p}{,} \PY{n}{phi}\PY{p}{]}\PY{p}{)}
\PY{n}{metric} \PY{o}{=} \PY{n}{sp}\PY{o}{.}\PY{n}{diag}\PY{p}{(}\PY{p}{(}\PY{l+m+mi}{1}\PY{o}{\PYZhy{}}\PY{n}{rs}\PY{o}{/}\PY{n}{r}\PY{p}{)}\PY{p}{,} \PY{o}{\PYZhy{}}\PY{l+m+mi}{1}\PY{o}{/}\PY{p}{(}\PY{l+m+mi}{1}\PY{o}{\PYZhy{}}\PY{n}{rs}\PY{o}{/}\PY{n}{r}\PY{p}{)}\PY{p}{,} \PY{o}{\PYZhy{}}\PY{n}{r}\PY{o}{*}\PY{o}{*}\PY{l+m+mi}{2}\PY{p}{,} \PY{o}{\PYZhy{}}\PY{n}{r}\PY{o}{*}\PY{o}{*}\PY{l+m+mi}{2}\PY{o}{*}\PY{p}{(}\PY{n}{sp}\PY{o}{.}\PY{n}{sin}\PY{p}{(}\PY{n}{theta}\PY{p}{)}\PY{p}{)}\PY{o}{*}\PY{o}{*}\PY{l+m+mi}{2}\PY{p}{)}
\PY{n}{geo} \PY{o}{=} \PY{n}{RiemannGeometry}\PY{p}{(}\PY{n}{metric}\PY{p}{,} \PY{n}{coords}\PY{p}{)}
\PY{n}{geo}\PY{o}{.}\PY{n}{calculate}\PY{p}{(}\PY{p}{)}
\PY{n}{geo}
\end{Verbatim}
\end{tcolorbox}

    \begin{Verbatim}[commandchars=\\\{\}]
Calculating {\ldots}
    \end{Verbatim}

    \begin{Verbatim}[commandchars=\\\{\}]
Christoffel Symbols : 100\%|\textblock\textblock\textblock\textblock\textblock\textblock\textblock\textblock\textblock\textblock\textblock\textblock\textblock\textblock\textblock\textblock\textblock\textblock\textblock\textblock\textblock\textblock\textblock\textblock\textblock\textblock\textblock\textblock\textblock\textblock\textblock\textblock\textblock\textblock\textblock| 4/4[00:00<00:00, 40.57it/s]
Riemann Tensor      : 100\%|\textblock\textblock\textblock\textblock\textblock\textblock\textblock\textblock\textblock\textblock\textblock\textblock\textblock\textblock\textblock\textblock\textblock\textblock\textblock\textblock\textblock\textblock\textblock\textblock\textblock\textblock\textblock\textblock\textblock\textblock\textblock\textblock\textblock\textblock\textblock| 4/4[00:00<00:00, 12.59it/s]
Ricci Tensor        : 100\%|\textblock\textblock\textblock\textblock\textblock\textblock\textblock\textblock\textblock\textblock\textblock\textblock\textblock\textblock\textblock\textblock\textblock\textblock\textblock\textblock\textblock\textblock\textblock\textblock\textblock\textblock\textblock\textblock\textblock\textblock\textblock\textblock\textblock| 4/4 [00:00<00:00,1763.79it/s]
Ricci Scalar        : 100\%|\textblock\textblock\textblock\textblock\textblock\textblock\textblock\textblock\textblock\textblock\textblock\textblock\textblock\textblock\textblock\textblock\textblock\textblock\textblock\textblock\textblock\textblock\textblock\textblock\textblock\textblock\textblock\textblock\textblock\textblock\textblock\textblock| 4/4 [00:00<00:00,22610.80it/s]
    \end{Verbatim}
 
            
\prompt{Out}{outcolor}{11}{}
    
    $$x^{a}=\begin{pmatrix}t\\r\\\theta\\\varphi\end{pmatrix},\quad g_{ab}=\begin{pmatrix} 1 - \frac{r_{s}}{r} & 0 & 0 & 0\\ 0 & - \frac{1}{1 - \frac{r_{s}}{r}} & 0 & 0\\ 0 & 0 & - r^{2} & 0\\ 0 & 0 & 0 & - r^{2} \sin^{2}{\left(\theta \right)} \end{pmatrix},\quad g^{ab}=\begin{pmatrix} \frac{1}{1 - \frac{r_{s}}{r}} & 0 & 0 & 0\\ 0 & -1 + \frac{r_{s}}{r} & 0 & 0\\ 0 & 0 & - \frac{1}{r^{2}} & 0\\ 0 & 0 & 0 & - \frac{1}{r^{2} \sin^{2}{\left(\theta \right)}} \end{pmatrix},$$
$${\Gamma^{a}}_{bc}=\begin{pmatrix} \begin{pmatrix} 0 & \frac{r_{s}}{2 r \left(r - r_{s}\right)} & 0 & 0\\ \frac{r_{s}}{2 r \left(r - r_{s}\right)} & 0 & 0 & 0\\ 0 & 0 & 0 & 0\\ 0 & 0 & 0 & 0 \end{pmatrix} & \begin{pmatrix} \frac{r_{s} \left(r - r_{s}\right)}{2 r^{3}} & 0 & 0 & 0\\ 0 & - \frac{r_{s}}{2 r \left(r - r_{s}\right)} & 0 & 0\\ 0 & 0 & - r + r_{s} & 0\\ 0 & 0 & 0 & \left(- r + r_{s}\right) \sin^{2}{\left(\theta \right)} \end{pmatrix} & \begin{pmatrix} 0 & 0 & 0 & 0\\ 0 & 0 & \frac{1}{r} & 0\\ 0 & \frac{1}{r} & 0 & 0\\ 0 & 0 & 0 & - \sin{\left(\theta \right)} \cos{\left(\theta \right)} \end{pmatrix} & \begin{pmatrix} 0 & 0 & 0 & 0\\ 0 & 0 & 0 & \frac{1}{r}\\ 0 & 0 & 0 & \frac{1}{\tan{\left(\theta \right)}}\\ 0 & \frac{1}{r} & \frac{1}{\tan{\left(\theta \right)}} & 0 \end{pmatrix} \end{pmatrix},$$
$${R^{a}}_{bcd}=\begin{pmatrix} \begin{pmatrix} 0 & 0 & 0 & 0\\ 0 & 0 & 0 & 0\\ 0 & 0 & 0 & 0\\ 0 & 0 & 0 & 0 \end{pmatrix} & \begin{pmatrix} 0 & \frac{r_{s}}{r^{2} \left(r - r_{s}\right)} & 0 & 0\\ - \frac{r_{s}}{r^{2} \left(r - r_{s}\right)} & 0 & 0 & 0\\ 0 & 0 & 0 & 0\\ 0 & 0 & 0 & 0 \end{pmatrix} & \begin{pmatrix} 0 & 0 & - \frac{r_{s}}{2 r} & 0\\ 0 & 0 & 0 & 0\\ \frac{r_{s}}{2 r} & 0 & 0 & 0\\ 0 & 0 & 0 & 0 \end{pmatrix} & \begin{pmatrix} 0 & 0 & 0 & - \frac{r_{s} \sin^{2}{\left(\theta \right)}}{2 r}\\ 0 & 0 & 0 & 0\\ 0 & 0 & 0 & 0\\ \frac{r_{s} \sin^{2}{\left(\theta \right)}}{2 r} & 0 & 0 & 0 \end{pmatrix}\\ \begin{pmatrix} 0 & \frac{r_{s} \left(r - r_{s}\right)}{r^{4}} & 0 & 0\\ \frac{r_{s} \left(- r + r_{s}\right)}{r^{4}} & 0 & 0 & 0\\ 0 & 0 & 0 & 0\\ 0 & 0 & 0 & 0 \end{pmatrix} & \begin{pmatrix} 0 & 0 & 0 & 0\\ 0 & 0 & 0 & 0\\ 0 & 0 & 0 & 0\\ 0 & 0 & 0 & 0 \end{pmatrix} & \begin{pmatrix} 0 & 0 & 0 & 0\\ 0 & 0 & - \frac{r_{s}}{2 r} & 0\\ 0 & \frac{r_{s}}{2 r} & 0 & 0\\ 0 & 0 & 0 & 0 \end{pmatrix} & \begin{pmatrix} 0 & 0 & 0 & 0\\ 0 & 0 & 0 & - \frac{r_{s} \sin^{2}{\left(\theta \right)}}{2 r}\\ 0 & 0 & 0 & 0\\ 0 & \frac{r_{s} \sin^{2}{\left(\theta \right)}}{2 r} & 0 & 0 \end{pmatrix}\\ \begin{pmatrix} 0 & 0 & \frac{r_{s} \left(- r + r_{s}\right)}{2 r^{4}} & 0\\ 0 & 0 & 0 & 0\\ \frac{r_{s} \left(r - r_{s}\right)}{2 r^{4}} & 0 & 0 & 0\\ 0 & 0 & 0 & 0 \end{pmatrix} & \begin{pmatrix} 0 & 0 & 0 & 0\\ 0 & 0 & \frac{r_{s}}{2 r^{2} \left(r - r_{s}\right)} & 0\\ 0 & - \frac{r_{s}}{2 r^{2} \left(r - r_{s}\right)} & 0 & 0\\ 0 & 0 & 0 & 0 \end{pmatrix} & \begin{pmatrix} 0 & 0 & 0 & 0\\ 0 & 0 & 0 & 0\\ 0 & 0 & 0 & 0\\ 0 & 0 & 0 & 0 \end{pmatrix} & \begin{pmatrix} 0 & 0 & 0 & 0\\ 0 & 0 & 0 & 0\\ 0 & 0 & 0 & \frac{r_{s} \sin^{2}{\left(\theta \right)}}{r}\\ 0 & 0 & - \frac{r_{s} \sin^{2}{\left(\theta \right)}}{r} & 0 \end{pmatrix}\\ \begin{pmatrix} 0 & 0 & 0 & \frac{r_{s} \left(- r + r_{s}\right)}{2 r^{4}}\\ 0 & 0 & 0 & 0\\ 0 & 0 & 0 & 0\\ \frac{r_{s} \left(r - r_{s}\right)}{2 r^{4}} & 0 & 0 & 0 \end{pmatrix} & \begin{pmatrix} 0 & 0 & 0 & 0\\ 0 & 0 & 0 & \frac{r_{s}}{2 r^{2} \left(r - r_{s}\right)}\\ 0 & 0 & 0 & 0\\ 0 & - \frac{r_{s}}{2 r^{2} \left(r - r_{s}\right)} & 0 & 0 \end{pmatrix} & \begin{pmatrix} 0 & 0 & 0 & 0\\ 0 & 0 & 0 & 0\\ 0 & 0 & 0 & - \frac{r_{s}}{r}\\ 0 & 0 & \frac{r_{s}}{r} & 0 \end{pmatrix} & \begin{pmatrix} 0 & 0 & 0 & 0\\ 0 & 0 & 0 & 0\\ 0 & 0 & 0 & 0\\ 0 & 0 & 0 & 0 \end{pmatrix} \end{pmatrix},$$
$$R_{ab}=\begin{pmatrix} 0 & 0 & 0 & 0\\ 0 & 0 & 0 & 0\\ 0 & 0 & 0 & 0\\ 0 & 0 & 0 & 0 \end{pmatrix},\quad R=0.$$

    \hypertarget{kerr}{%
\paragraph{Kerr}\label{kerr}}

One of the solution for Einstein field equations, which corresponds to
the gravitational field of a uncharged, rotating, spherically symmetric
body, is Kerr metric, and the line element is given by \[
ds^2 = -\left(1-\frac{r_{s}r}{\Sigma}\right)dt^2+\frac{\Sigma}{\Delta}dr^2+\Sigma d\theta^2+\left(r^2+a^2+\frac{r_{s}ra^2}{\Sigma}\sin^2\theta \right)\sin^2\theta \ d\phi^2-{\frac{2r_{s}ra\sin^2\theta}{\Sigma}}\,dt\,d\phi
\] which is in \emph{Boyer-Lindquist coordinates}, then we can calculate
the \textbf{\emph{Christoffel symbols}}, \textbf{\emph{Riemann tensor}},
\textbf{\emph{Ricci tensor}} and \textbf{\emph{Ricci scalar}}.

\begin{quote}
Here using Geometrized unit system \(G=c=1\).
\end{quote}

    \begin{tcolorbox}[breakable, size=fbox, boxrule=1pt, pad at break*=1mm,colback=cellbackground, colframe=cellborder]
\prompt{In}{incolor}{24}{\boxspacing}
\begin{Verbatim}[commandchars=\\\{\}]
\PY{n}{t}\PY{p}{,} \PY{n}{r}\PY{p}{,} \PY{n}{theta}\PY{p}{,} \PY{n}{phi} \PY{o}{=} \PY{n}{sp}\PY{o}{.}\PY{n}{symbols}\PY{p}{(}\PY{l+s+s1}{\PYZsq{}}\PY{l+s+s1}{t r theta varphi}\PY{l+s+s1}{\PYZsq{}}\PY{p}{)}
\PY{n}{rs}\PY{p}{,} \PY{n}{J}\PY{p}{,} \PY{n}{M} \PY{o}{=} \PY{n}{sp}\PY{o}{.}\PY{n}{symbols}\PY{p}{(}\PY{l+s+s1}{\PYZsq{}}\PY{l+s+s1}{r\PYZus{}s J, M}\PY{l+s+s1}{\PYZsq{}}\PY{p}{)}
\PY{n}{coords} \PY{o}{=} \PY{n}{sp}\PY{o}{.}\PY{n}{Matrix}\PY{p}{(}\PY{p}{[}\PY{n}{t}\PY{p}{,} \PY{n}{r}\PY{p}{,} \PY{n}{theta}\PY{p}{,} \PY{n}{phi}\PY{p}{]}\PY{p}{)}
\PY{n}{a} \PY{o}{=} \PY{n}{J}\PY{o}{/}\PY{n}{M}
\PY{n}{mu} \PY{o}{=} \PY{n}{r}\PY{o}{*}\PY{o}{*}\PY{l+m+mi}{2} \PY{o}{+} \PY{n}{a}\PY{o}{*}\PY{o}{*}\PY{l+m+mi}{2}
\PY{n}{Sigma} \PY{o}{=} \PY{n}{r}\PY{o}{*}\PY{o}{*}\PY{l+m+mi}{2} \PY{o}{+} \PY{n}{a}\PY{o}{*}\PY{o}{*}\PY{l+m+mi}{2} \PY{o}{*} \PY{n}{sp}\PY{o}{.}\PY{n}{cos}\PY{p}{(}\PY{n}{theta}\PY{p}{)}\PY{o}{*}\PY{o}{*}\PY{l+m+mi}{2}
\PY{n}{Delta} \PY{o}{=} \PY{n}{r}\PY{o}{*}\PY{o}{*}\PY{l+m+mi}{2} \PY{o}{+} \PY{n}{a}\PY{o}{*}\PY{o}{*}\PY{l+m+mi}{2} \PY{o}{\PYZhy{}} \PY{n}{rs}\PY{o}{*}\PY{n}{r}

\PY{n}{metric} \PY{o}{=} \PY{n}{sp}\PY{o}{.}\PY{n}{Matrix}\PY{p}{(}\PY{p}{[}
    \PY{p}{[}\PY{o}{\PYZhy{}}\PY{p}{(}\PY{l+m+mi}{1}\PY{o}{\PYZhy{}}\PY{n}{rs}\PY{o}{*}\PY{n}{r}\PY{o}{/}\PY{n}{Sigma}\PY{p}{)}\PY{p}{,}             \PY{l+m+mi}{0}\PY{p}{,}           \PY{l+m+mi}{0}\PY{p}{,}     \PY{o}{\PYZhy{}}\PY{n}{rs}\PY{o}{*}\PY{n}{r}\PY{o}{*}\PY{n}{a}\PY{o}{*}\PY{n}{sp}\PY{o}{.}\PY{n}{sin}\PY{p}{(}\PY{n}{theta}\PY{p}{)}\PY{o}{*}\PY{o}{*}\PY{l+m+mi}{2}\PY{o}{/}\PY{n}{Sigma}\PY{p}{]}\PY{p}{,}
    \PY{p}{[}\PY{l+m+mi}{0}\PY{p}{,}                           \PY{n}{Sigma}\PY{o}{/}\PY{n}{Delta}\PY{p}{,} \PY{l+m+mi}{0}\PY{p}{,}     \PY{l+m+mi}{0}\PY{p}{]}\PY{p}{,}
    \PY{p}{[}\PY{l+m+mi}{0}\PY{p}{,}                           \PY{l+m+mi}{0}\PY{p}{,}           \PY{n}{Sigma}\PY{p}{,} \PY{l+m+mi}{0}\PY{p}{]}\PY{p}{,}
    \PY{p}{[}\PY{o}{\PYZhy{}}\PY{n}{rs}\PY{o}{*}\PY{n}{r}\PY{o}{*}\PY{n}{a}\PY{o}{*}\PY{n}{sp}\PY{o}{.}\PY{n}{sin}\PY{p}{(}\PY{n}{theta}\PY{p}{)}\PY{o}{*}\PY{o}{*}\PY{l+m+mi}{2}\PY{o}{/}\PY{n}{Sigma}\PY{p}{,} \PY{l+m+mi}{0}\PY{p}{,}           \PY{l+m+mi}{0}\PY{p}{,}     \PY{p}{(}\PY{n}{mu}\PY{o}{+}\PY{n}{rs}\PY{o}{*}\PY{n}{r}\PY{o}{*}\PY{n}{a}\PY{o}{*}\PY{o}{*}\PY{l+m+mi}{2}\PY{o}{*}\PY{n}{sp}\PY{o}{.}\PY{n}{sin}\PY{p}{(}\PY{n}{theta}\PY{p}{)}\PY{o}{*}\PY{o}{*}\PY{l+m+mi}{2}\PY{o}{/}\PY{n}{Sigma}\PY{p}{)}\PY{o}{*}\PY{n}{sp}\PY{o}{.}\PY{n}{sin}\PY{p}{(}\PY{n}{theta}\PY{p}{)}\PY{o}{*}\PY{o}{*}\PY{l+m+mi}{2}\PY{p}{]}
\PY{p}{]}\PY{p}{)}
\PY{c+c1}{\PYZsh{} WARM: !!! It would take many time to calcuate!!!!!}
\PY{c+c1}{\PYZsh{} geo = RiemannGeometry(metric, coords)}
\PY{c+c1}{\PYZsh{} geo.calculate()}
\PY{c+c1}{\PYZsh{} geo}
\end{Verbatim}
\end{tcolorbox}

    \hypertarget{reissner-nordstruxf6m}{%
\paragraph{Reissner-Nordström}\label{reissner-nordstruxf6m}}

One of the solution for Einstein field equations, which corresponds to
the gravitational field of a charged, non-rotating, spherically
symmetric body, is Weak-Field metric, and the line element is given by
\[
ds^2 = \left(1 - \frac{r_s}{r} + \frac{r_Q^2}{r^2}\right)dt^2 - \left(1 - \frac{r_s}{r} + \frac{r_Q^2}{r^2}\right) dr^2 - d\theta^2 - r^2 \sin^2\theta d\varphi,
\] where \(\Phi\) is a function of \(r\) (usual Newtonian gravitational
potential \(\Phi=-GM/r\)), then we can calculate the
\textbf{\emph{Christoffel symbols}}, \textbf{\emph{Riemann tensor}},
\textbf{\emph{Ricci tensor}} and \textbf{\emph{Ricci scalar}}.

\begin{quote}
Here using Geometrized unit system \(G=c=1\).
\end{quote}

    \begin{tcolorbox}[breakable, size=fbox, boxrule=1pt, pad at break*=1mm,colback=cellbackground, colframe=cellborder]
\prompt{In}{incolor}{12}{\boxspacing}
\begin{Verbatim}[commandchars=\\\{\}]
\PY{n}{t}\PY{p}{,} \PY{n}{r}\PY{p}{,} \PY{n}{theta}\PY{p}{,} \PY{n}{phi} \PY{o}{=} \PY{n}{sp}\PY{o}{.}\PY{n}{symbols}\PY{p}{(}\PY{l+s+s1}{\PYZsq{}}\PY{l+s+s1}{t r theta varphi}\PY{l+s+s1}{\PYZsq{}}\PY{p}{)}
\PY{n}{rs}\PY{p}{,} \PY{n}{rQ} \PY{o}{=} \PY{n}{sp}\PY{o}{.}\PY{n}{symbols}\PY{p}{(}\PY{l+s+s1}{\PYZsq{}}\PY{l+s+s1}{r\PYZus{}s r\PYZus{}Q}\PY{l+s+s1}{\PYZsq{}}\PY{p}{)}
\PY{n}{coords} \PY{o}{=} \PY{n}{sp}\PY{o}{.}\PY{n}{Matrix}\PY{p}{(}\PY{p}{[}\PY{n}{t}\PY{p}{,} \PY{n}{r}\PY{p}{,} \PY{n}{theta}\PY{p}{,} \PY{n}{phi}\PY{p}{]}\PY{p}{)}
\PY{n}{metric} \PY{o}{=} \PY{n}{sp}\PY{o}{.}\PY{n}{diag}\PY{p}{(}\PY{p}{(}\PY{l+m+mi}{1}\PY{o}{\PYZhy{}}\PY{n}{rs}\PY{o}{/}\PY{n}{r} \PY{o}{+} \PY{n}{rQ}\PY{o}{*}\PY{o}{*}\PY{l+m+mi}{2}\PY{o}{/}\PY{n}{r}\PY{o}{*}\PY{o}{*}\PY{l+m+mi}{2}\PY{p}{)}\PY{p}{,} \PY{o}{\PYZhy{}}\PY{l+m+mi}{1}\PY{o}{/}\PY{p}{(}\PY{l+m+mi}{1}\PY{o}{\PYZhy{}}\PY{n}{rs}\PY{o}{/}\PY{n}{r}\PY{o}{+} \PY{n}{rQ}\PY{o}{*}\PY{o}{*}\PY{l+m+mi}{2}\PY{o}{/}\PY{n}{r}\PY{o}{*}\PY{o}{*}\PY{l+m+mi}{2}\PY{p}{)}\PY{p}{,} \PY{o}{\PYZhy{}}\PY{n}{r}\PY{o}{*}\PY{o}{*}\PY{l+m+mi}{2}\PY{p}{,} \PY{o}{\PYZhy{}}\PY{n}{r}\PY{o}{*}\PY{o}{*}\PY{l+m+mi}{2}\PY{o}{*}\PY{p}{(}\PY{n}{sp}\PY{o}{.}\PY{n}{sin}\PY{p}{(}\PY{n}{theta}\PY{p}{)}\PY{p}{)}\PY{o}{*}\PY{o}{*}\PY{l+m+mi}{2}\PY{p}{)}
\PY{n}{geo} \PY{o}{=} \PY{n}{RiemannGeometry}\PY{p}{(}\PY{n}{metric}\PY{p}{,} \PY{n}{coords}\PY{p}{)}
\PY{n}{geo}\PY{o}{.}\PY{n}{calculate}\PY{p}{(}\PY{p}{)}
\PY{n}{geo}
\end{Verbatim}
\end{tcolorbox}

    \begin{Verbatim}[commandchars=\\\{\}]
Calculating {\ldots}
    \end{Verbatim}

    \begin{Verbatim}[commandchars=\\\{\}]
Christoffel Symbols : 100\%|\textblock\textblock\textblock\textblock\textblock\textblock\textblock\textblock\textblock\textblock\textblock\textblock\textblock\textblock\textblock\textblock\textblock\textblock\textblock\textblock\textblock\textblock\textblock\textblock\textblock\textblock\textblock\textblock\textblock\textblock\textblock\textblock\textblock\textblock\textblock| 4/4[00:00<00:00, 22.81it/s]
Riemann Tensor      : 100\%|\textblock\textblock\textblock\textblock\textblock\textblock\textblock\textblock\textblock\textblock\textblock\textblock\textblock\textblock\textblock\textblock\textblock\textblock\textblock\textblock\textblock\textblock\textblock\textblock\textblock\textblock\textblock\textblock\textblock\textblock\textblock\textblock\textblock\textblock\textblock| 4/4[00:00<00:00,  4.55it/s]
Ricci Tensor        : 100\%|\textblock\textblock\textblock\textblock\textblock\textblock\textblock\textblock\textblock\textblock\textblock\textblock\textblock\textblock\textblock\textblock\textblock\textblock\textblock\textblock\textblock\textblock\textblock\textblock\textblock\textblock\textblock\textblock\textblock\textblock\textblock\textblock\textblock\textblock\textblock| 4/4[00:00<00:00, 59.79it/s]
Ricci Scalar        : 100\%|\textblock\textblock\textblock\textblock\textblock\textblock\textblock\textblock\textblock\textblock\textblock\textblock\textblock\textblock\textblock\textblock\textblock\textblock\textblock\textblock\textblock\textblock\textblock\textblock\textblock\textblock\textblock\textblock\textblock\textblock\textblock\textblock| 4/4 [00:00<00:00,
11732.32it/s]
    \end{Verbatim}
 
            
\prompt{Out}{outcolor}{12}{}
    
    $$x^{a}=\begin{pmatrix}t\\r\\\theta\\\varphi\end{pmatrix},\quad g_{ab}=\begin{pmatrix} 1 - \frac{r_{s}}{r} + \frac{r_{Q}^{2}}{r^{2}} & 0 & 0 & 0\\ 0 & - \frac{1}{1 - \frac{r_{s}}{r} + \frac{r_{Q}^{2}}{r^{2}}} & 0 & 0\\ 0 & 0 & - r^{2} & 0\\ 0 & 0 & 0 & - r^{2} \sin^{2}{\left(\theta \right)} \end{pmatrix},\quad g^{ab}=\begin{pmatrix} \frac{r^{2}}{r^{2} - r r_{s} + r_{Q}^{2}} & 0 & 0 & 0\\ 0 & \frac{- r^{2} + r r_{s} - r_{Q}^{2}}{r^{2}} & 0 & 0\\ 0 & 0 & - \frac{1}{r^{2}} & 0\\ 0 & 0 & 0 & - \frac{1}{r^{2} \sin^{2}{\left(\theta \right)}} \end{pmatrix},$$
$${\Gamma^{a}}_{bc}=\begin{pmatrix} \begin{pmatrix} 0 & \frac{\frac{r r_{s}}{2} - r_{Q}^{2}}{r \left(r^{2} - r r_{s} + r_{Q}^{2}\right)} & 0 & 0\\ \frac{\frac{r r_{s}}{2} - r_{Q}^{2}}{r \left(r^{2} - r r_{s} + r_{Q}^{2}\right)} & 0 & 0 & 0\\ 0 & 0 & 0 & 0\\ 0 & 0 & 0 & 0 \end{pmatrix} & \begin{pmatrix} \frac{\left(r r_{s} - 2 r_{Q}^{2}\right) \left(r^{2} - r r_{s} + r_{Q}^{2}\right)}{2 r^{5}} & 0 & 0 & 0\\ 0 & \frac{- \frac{r r_{s}}{2} + r_{Q}^{2}}{r \left(r^{2} - r r_{s} + r_{Q}^{2}\right)} & 0 & 0\\ 0 & 0 & - r + r_{s} - \frac{r_{Q}^{2}}{r} & 0\\ 0 & 0 & 0 & \frac{\left(- r^{2} + r r_{s} - r_{Q}^{2}\right) \sin^{2}{\left(\theta \right)}}{r} \end{pmatrix} & \begin{pmatrix} 0 & 0 & 0 & 0\\ 0 & 0 & \frac{1}{r} & 0\\ 0 & \frac{1}{r} & 0 & 0\\ 0 & 0 & 0 & - \sin{\left(\theta \right)} \cos{\left(\theta \right)} \end{pmatrix} & \begin{pmatrix} 0 & 0 & 0 & 0\\ 0 & 0 & 0 & \frac{1}{r}\\ 0 & 0 & 0 & \frac{1}{\tan{\left(\theta \right)}}\\ 0 & \frac{1}{r} & \frac{1}{\tan{\left(\theta \right)}} & 0 \end{pmatrix} \end{pmatrix},$$
$${R^{a}}_{bcd}=\begin{pmatrix} \begin{pmatrix} 0 & 0 & 0 & 0\\ 0 & 0 & 0 & 0\\ 0 & 0 & 0 & 0\\ 0 & 0 & 0 & 0 \end{pmatrix} & \begin{pmatrix} 0 & \frac{r r_{s} - 3 r_{Q}^{2}}{r^{2} \left(r^{2} - r r_{s} + r_{Q}^{2}\right)} & 0 & 0\\ \frac{- r r_{s} + 3 r_{Q}^{2}}{r^{2} \left(r^{2} - r r_{s} + r_{Q}^{2}\right)} & 0 & 0 & 0\\ 0 & 0 & 0 & 0\\ 0 & 0 & 0 & 0 \end{pmatrix} & \begin{pmatrix} 0 & 0 & \frac{- \frac{r r_{s}}{2} + r_{Q}^{2}}{r^{2}} & 0\\ 0 & 0 & 0 & 0\\ \frac{\frac{r r_{s}}{2} - r_{Q}^{2}}{r^{2}} & 0 & 0 & 0\\ 0 & 0 & 0 & 0 \end{pmatrix} & \begin{pmatrix} 0 & 0 & 0 & \frac{\left(- \frac{r r_{s}}{2} + r_{Q}^{2}\right) \sin^{2}{\left(\theta \right)}}{r^{2}}\\ 0 & 0 & 0 & 0\\ 0 & 0 & 0 & 0\\ \frac{\left(\frac{r r_{s}}{2} - r_{Q}^{2}\right) \sin^{2}{\left(\theta \right)}}{r^{2}} & 0 & 0 & 0 \end{pmatrix}\\ \begin{pmatrix} 0 & \frac{r^{3} r_{s} - 3 r^{2} r_{Q}^{2} - r^{2} r_{s}^{2} + 4 r r_{Q}^{2} r_{s} - 3 r_{Q}^{4}}{r^{6}} & 0 & 0\\ \frac{- r^{3} r_{s} + 3 r^{2} r_{Q}^{2} + r^{2} r_{s}^{2} - 4 r r_{Q}^{2} r_{s} + 3 r_{Q}^{4}}{r^{6}} & 0 & 0 & 0\\ 0 & 0 & 0 & 0\\ 0 & 0 & 0 & 0 \end{pmatrix} & \begin{pmatrix} 0 & 0 & 0 & 0\\ 0 & 0 & 0 & 0\\ 0 & 0 & 0 & 0\\ 0 & 0 & 0 & 0 \end{pmatrix} & \begin{pmatrix} 0 & 0 & 0 & 0\\ 0 & 0 & \frac{- \frac{r r_{s}}{2} + r_{Q}^{2}}{r^{2}} & 0\\ 0 & \frac{\frac{r r_{s}}{2} - r_{Q}^{2}}{r^{2}} & 0 & 0\\ 0 & 0 & 0 & 0 \end{pmatrix} & \begin{pmatrix} 0 & 0 & 0 & 0\\ 0 & 0 & 0 & \frac{\left(- \frac{r r_{s}}{2} + r_{Q}^{2}\right) \sin^{2}{\left(\theta \right)}}{r^{2}}\\ 0 & 0 & 0 & 0\\ 0 & \frac{\left(\frac{r r_{s}}{2} - r_{Q}^{2}\right) \sin^{2}{\left(\theta \right)}}{r^{2}} & 0 & 0 \end{pmatrix}\\ \begin{pmatrix} 0 & 0 & - \frac{\left(r r_{s} - 2 r_{Q}^{2}\right) \left(r^{2} - r r_{s} + r_{Q}^{2}\right)}{2 r^{6}} & 0\\ 0 & 0 & 0 & 0\\ \frac{\left(r r_{s} - 2 r_{Q}^{2}\right) \left(r^{2} - r r_{s} + r_{Q}^{2}\right)}{2 r^{6}} & 0 & 0 & 0\\ 0 & 0 & 0 & 0 \end{pmatrix} & \begin{pmatrix} 0 & 0 & 0 & 0\\ 0 & 0 & \frac{\frac{r r_{s}}{2} - r_{Q}^{2}}{r^{2} \left(r^{2} - r r_{s} + r_{Q}^{2}\right)} & 0\\ 0 & \frac{- \frac{r r_{s}}{2} + r_{Q}^{2}}{r^{2} \left(r^{2} - r r_{s} + r_{Q}^{2}\right)} & 0 & 0\\ 0 & 0 & 0 & 0 \end{pmatrix} & \begin{pmatrix} 0 & 0 & 0 & 0\\ 0 & 0 & 0 & 0\\ 0 & 0 & 0 & 0\\ 0 & 0 & 0 & 0 \end{pmatrix} & \begin{pmatrix} 0 & 0 & 0 & 0\\ 0 & 0 & 0 & 0\\ 0 & 0 & 0 & \frac{\left(r r_{s} - r_{Q}^{2}\right) \sin^{2}{\left(\theta \right)}}{r^{2}}\\ 0 & 0 & \frac{\left(- r r_{s} + r_{Q}^{2}\right) \sin^{2}{\left(\theta \right)}}{r^{2}} & 0 \end{pmatrix}\\ \begin{pmatrix} 0 & 0 & 0 & - \frac{\left(r r_{s} - 2 r_{Q}^{2}\right) \left(r^{2} - r r_{s} + r_{Q}^{2}\right)}{2 r^{6}}\\ 0 & 0 & 0 & 0\\ 0 & 0 & 0 & 0\\ \frac{\left(r r_{s} - 2 r_{Q}^{2}\right) \left(r^{2} - r r_{s} + r_{Q}^{2}\right)}{2 r^{6}} & 0 & 0 & 0 \end{pmatrix} & \begin{pmatrix} 0 & 0 & 0 & 0\\ 0 & 0 & 0 & \frac{\frac{r r_{s}}{2} - r_{Q}^{2}}{r^{2} \left(r^{2} - r r_{s} + r_{Q}^{2}\right)}\\ 0 & 0 & 0 & 0\\ 0 & \frac{- \frac{r r_{s}}{2} + r_{Q}^{2}}{r^{2} \left(r^{2} - r r_{s} + r_{Q}^{2}\right)} & 0 & 0 \end{pmatrix} & \begin{pmatrix} 0 & 0 & 0 & 0\\ 0 & 0 & 0 & 0\\ 0 & 0 & 0 & \frac{- r r_{s} + r_{Q}^{2}}{r^{2}}\\ 0 & 0 & \frac{r r_{s} - r_{Q}^{2}}{r^{2}} & 0 \end{pmatrix} & \begin{pmatrix} 0 & 0 & 0 & 0\\ 0 & 0 & 0 & 0\\ 0 & 0 & 0 & 0\\ 0 & 0 & 0 & 0 \end{pmatrix} \end{pmatrix},$$
$$R_{ab}=\begin{pmatrix} \frac{r_{Q}^{2} \left(r^{2} - r r_{s} + r_{Q}^{2}\right)}{r^{6}} & 0 & 0 & 0\\ 0 & - \frac{r_{Q}^{2}}{r^{2} \left(r^{2} - r r_{s} + r_{Q}^{2}\right)} & 0 & 0\\ 0 & 0 & \frac{r_{Q}^{2}}{r^{2}} & 0\\ 0 & 0 & 0 & \frac{r_{Q}^{2} \sin^{2}{\left(\theta \right)}}{r^{2}} \end{pmatrix},\quad R=0.$$

\newpage %%%%%%%%%%%%%%%%%%%%%%%%%%%%%%%%%%%%%%%%%%%%%%%%%%%%%%%% 
\begin{landscape}
    \hypertarget{weak-field}{%
\paragraph{Weak-Field}\label{weak-field}}

One of the solution for Einstein field equations in the limit of weak
gravity, is Weak-Field metric, where the line element is given by \[
ds^2 = \left(1 + 2\Phi^2(r)\right)dt^2 - \left(1 + 2\Phi^2(r)\right) dr^2 - \left(1 + 2\Phi^2(r)\right) d\theta^2 - \left(1 + 2\Phi^2(r)\right) r^2 \sin^2\theta d\varphi,
\] where \(\Phi\) is a function of \(r\) (usual Newtonian gravitational
potential \(\Phi=-GM/r\)), then we can calculate the
\textbf{\emph{Christoffel symbols}}, \textbf{\emph{Riemann tensor}},
\textbf{\emph{Ricci tensor}} and \textbf{\emph{Ricci scalar}}.

\begin{quote}
Here using Geometrized unit system \(G=c=1\).
\end{quote}

    \begin{tcolorbox}[breakable, size=fbox, boxrule=1pt, pad at break*=1mm,colback=cellbackground, colframe=cellborder]
\prompt{In}{incolor}{25}{\boxspacing}
\begin{Verbatim}[commandchars=\\\{\}]
\PY{c+c1}{\PYZsh{} coordinates }
\PY{n}{t}\PY{p}{,} \PY{n}{r}\PY{p}{,} \PY{n}{theta}\PY{p}{,} \PY{n}{phi} \PY{o}{=} \PY{n}{sp}\PY{o}{.}\PY{n}{symbols}\PY{p}{(}\PY{l+s+s1}{\PYZsq{}}\PY{l+s+s1}{t r theta varphi}\PY{l+s+s1}{\PYZsq{}}\PY{p}{)}
\PY{c+c1}{\PYZsh{} dependent function}
\PY{n}{Phi} \PY{o}{=} \PY{n}{sp}\PY{o}{.}\PY{n}{Function}\PY{p}{(}\PY{l+s+s1}{\PYZsq{}}\PY{l+s+s1}{Phi}\PY{l+s+s1}{\PYZsq{}}\PY{p}{)}\PY{p}{(}\PY{n}{r}\PY{p}{)}

\PY{c+c1}{\PYZsh{} M = sp.symbols(\PYZsq{}M\PYZsq{})}
\PY{c+c1}{\PYZsh{} Phi = \PYZhy{}M/r}
\PY{c+c1}{\PYZsh{} \PYZhy{}\PYZhy{}\PYZhy{}\PYZhy{}\PYZhy{}\PYZhy{}\PYZhy{}\PYZhy{}\PYZhy{}\PYZhy{}\PYZhy{}\PYZhy{}\PYZhy{}\PYZhy{}\PYZhy{}\PYZhy{}\PYZhy{}\PYZhy{}\PYZhy{}\PYZhy{}\PYZhy{}\PYZhy{}\PYZhy{}\PYZhy{}\PYZhy{}\PYZhy{}\PYZhy{}\PYZhy{}\PYZhy{}\PYZhy{}}
\PY{n}{coords} \PY{o}{=} \PY{n}{sp}\PY{o}{.}\PY{n}{Matrix}\PY{p}{(}\PY{p}{[}\PY{n}{t}\PY{p}{,} \PY{n}{r}\PY{p}{,} \PY{n}{theta}\PY{p}{,} \PY{n}{phi}\PY{p}{]}\PY{p}{)}
\PY{n}{metrix} \PY{o}{=} \PY{n}{sp}\PY{o}{.}\PY{n}{diag}\PY{p}{(}\PY{o}{\PYZhy{}}\PY{p}{(}\PY{l+m+mi}{1}\PY{o}{+}\PY{l+m+mi}{2}\PY{o}{*}\PY{n}{Phi}\PY{p}{)}\PY{p}{,} \PY{o}{+}\PY{p}{(}\PY{l+m+mi}{1}\PY{o}{\PYZhy{}}\PY{l+m+mi}{2}\PY{o}{*}\PY{n}{Phi}\PY{p}{)}\PY{p}{,} \PY{o}{+}\PY{p}{(}\PY{l+m+mi}{1}\PY{o}{\PYZhy{}}\PY{l+m+mi}{2}\PY{o}{*}\PY{n}{Phi}\PY{p}{)}\PY{o}{*}\PY{n}{r}\PY{o}{*}\PY{o}{*}\PY{l+m+mi}{2}\PY{p}{,} \PY{o}{+}\PY{p}{(}\PY{l+m+mi}{1}\PY{o}{\PYZhy{}}\PY{l+m+mi}{2}\PY{o}{*}\PY{n}{Phi}\PY{p}{)}\PY{o}{*}\PY{n}{r}\PY{o}{*}\PY{o}{*}\PY{l+m+mi}{2}\PY{o}{*}\PY{p}{(}\PY{n}{sp}\PY{o}{.}\PY{n}{sin}\PY{p}{(}\PY{n}{theta}\PY{p}{)}\PY{p}{)}\PY{o}{*}\PY{o}{*}\PY{l+m+mi}{2}\PY{p}{)}

\PY{c+c1}{\PYZsh{} WARM: !!! It would take many time to calcuate!!!!!}
\PY{n}{geo} \PY{o}{=} \PY{n}{RiemannGeometry}\PY{p}{(}\PY{n}{metrix}\PY{p}{,} \PY{n}{coords}\PY{p}{)}
\PY{n}{geo}\PY{o}{.}\PY{n}{calculate}\PY{p}{(}\PY{p}{)}
\PY{n}{geo}
\end{Verbatim}
\end{tcolorbox}

    \begin{Verbatim}[commandchars=\\\{\}]
Calculating {\ldots}
    \end{Verbatim}

    \begin{Verbatim}[commandchars=\\\{\}]
Christoffel Symbols : 100\%|\textblock\textblock\textblock\textblock\textblock\textblock\textblock\textblock\textblock\textblock\textblock\textblock\textblock\textblock\textblock\textblock\textblock\textblock\textblock\textblock\textblock\textblock\textblock\textblock\textblock\textblock\textblock\textblock\textblock\textblock\textblock\textblock\textblock\textblock\textblock| 4/4[00:00<00:00, 13.07it/s]
Riemann Tensor      : 100\%|\textblock\textblock\textblock\textblock\textblock\textblock\textblock\textblock\textblock\textblock\textblock\textblock\textblock\textblock\textblock\textblock\textblock\textblock\textblock\textblock\textblock\textblock\textblock\textblock\textblock\textblock\textblock\textblock\textblock\textblock\textblock\textblock\textblock\textblock\textblock| 4/4[00:01<00:00,  3.22it/s]
Ricci Tensor        : 100\%|\textblock\textblock\textblock\textblock\textblock\textblock\textblock\textblock\textblock\textblock\textblock\textblock\textblock\textblock\textblock\textblock\textblock\textblock\textblock\textblock\textblock\textblock\textblock\textblock\textblock\textblock\textblock\textblock\textblock\textblock\textblock\textblock\textblock\textblock\textblock| 4/4[00:00<00:00,  9.54it/s]
Ricci Scalar        : 100\%|\textblock\textblock\textblock\textblock\textblock\textblock\textblock\textblock\textblock\textblock\textblock\textblock\textblock\textblock\textblock\textblock\textblock\textblock\textblock\textblock\textblock\textblock\textblock\textblock\textblock\textblock\textblock\textblock\textblock\textblock\textblock\textblock| 4/4 [00:00<00:00,
12087.33it/s]
    \end{Verbatim}
 
            
\prompt{Out}{outcolor}{25}{}
    
    $$x^{a}=\begin{pmatrix}t\\r\\\theta\\\varphi\end{pmatrix},\quad g_{ab}=\begin{pmatrix} - 2 \Phi{\left(r \right)} - 1 & 0 & 0 & 0\\ 0 & 1 - 2 \Phi{\left(r \right)} & 0 & 0\\ 0 & 0 & r^{2} \cdot \left(1 - 2 \Phi{\left(r \right)}\right) & 0\\ 0 & 0 & 0 & r^{2} \cdot \left(1 - 2 \Phi{\left(r \right)}\right) \sin^{2}{\left(\theta \right)} \end{pmatrix},\quad g^{ab}=\begin{pmatrix} \frac{1}{- 2 \Phi{\left(r \right)} - 1} & 0 & 0 & 0\\ 0 & \frac{1}{1 - 2 \Phi{\left(r \right)}} & 0 & 0\\ 0 & 0 & - \frac{1}{2 r^{2} \Phi{\left(r \right)} - r^{2}} & 0\\ 0 & 0 & 0 & - \frac{1}{2 r^{2} \Phi{\left(r \right)} \sin^{2}{\left(\theta \right)} - r^{2} \sin^{2}{\left(\theta \right)}} \end{pmatrix},$$
$${\Gamma^{a}}_{bc}=\begin{pmatrix} \begin{pmatrix} 0 & \frac{\frac{d}{d r} \Phi{\left(r \right)}}{2 \Phi{\left(r \right)} + 1} & 0 & 0\\ \frac{\frac{d}{d r} \Phi{\left(r \right)}}{2 \Phi{\left(r \right)} + 1} & 0 & 0 & 0\\ 0 & 0 & 0 & 0\\ 0 & 0 & 0 & 0 \end{pmatrix} & \begin{pmatrix} - \frac{\frac{d}{d r} \Phi{\left(r \right)}}{2 \Phi{\left(r \right)} - 1} & 0 & 0 & 0\\ 0 & \frac{\frac{d}{d r} \Phi{\left(r \right)}}{2 \Phi{\left(r \right)} - 1} & 0 & 0\\ 0 & 0 & \frac{r \left(- r \frac{d}{d r} \Phi{\left(r \right)} - 2 \Phi{\left(r \right)} + 1\right)}{2 \Phi{\left(r \right)} - 1} & 0\\ 0 & 0 & 0 & \frac{r \left(- r \frac{d}{d r} \Phi{\left(r \right)} - 2 \Phi{\left(r \right)} + 1\right) \sin^{2}{\left(\theta \right)}}{2 \Phi{\left(r \right)} - 1} \end{pmatrix} & \begin{pmatrix} 0 & 0 & 0 & 0\\ 0 & 0 & \frac{r \frac{d}{d r} \Phi{\left(r \right)} + 2 \Phi{\left(r \right)} - 1}{r \left(2 \Phi{\left(r \right)} - 1\right)} & 0\\ 0 & \frac{r \frac{d}{d r} \Phi{\left(r \right)} + 2 \Phi{\left(r \right)} - 1}{r \left(2 \Phi{\left(r \right)} - 1\right)} & 0 & 0\\ 0 & 0 & 0 & - \sin{\left(\theta \right)} \cos{\left(\theta \right)} \end{pmatrix} & \begin{pmatrix} 0 & 0 & 0 & 0\\ 0 & 0 & 0 & \frac{r \frac{d}{d r} \Phi{\left(r \right)} + 2 \Phi{\left(r \right)} - 1}{r \left(2 \Phi{\left(r \right)} - 1\right)}\\ 0 & 0 & 0 & \frac{1}{\tan{\left(\theta \right)}}\\ 0 & \frac{r \frac{d}{d r} \Phi{\left(r \right)} + 2 \Phi{\left(r \right)} - 1}{r \left(2 \Phi{\left(r \right)} - 1\right)} & \frac{1}{\tan{\left(\theta \right)}} & 0 \end{pmatrix} \end{pmatrix},$$
$${R^{a}}_{bcd}=\begin{pmatrix} \begin{pmatrix} 0 & 0 & 0 & 0\\ 0 & 0 & 0 & 0\\ 0 & 0 & 0 & 0\\ 0 & 0 & 0 & 0 \end{pmatrix} & \begin{pmatrix} 0 & \frac{- 4 \Phi^{2}{\left(r \right)} \frac{d^{2}}{d r^{2}} \Phi{\left(r \right)} + 4 \Phi{\left(r \right)} \left(\frac{d}{d r} \Phi{\left(r \right)}\right)^{2} + \frac{d^{2}}{d r^{2}} \Phi{\left(r \right)}}{8 \Phi^{3}{\left(r \right)} + 4 \Phi^{2}{\left(r \right)} - 2 \Phi{\left(r \right)} - 1} & 0 & 0\\ \frac{4 \Phi^{2}{\left(r \right)} \frac{d^{2}}{d r^{2}} \Phi{\left(r \right)} - 4 \Phi{\left(r \right)} \left(\frac{d}{d r} \Phi{\left(r \right)}\right)^{2} - \frac{d^{2}}{d r^{2}} \Phi{\left(r \right)}}{8 \Phi^{3}{\left(r \right)} + 4 \Phi^{2}{\left(r \right)} - 2 \Phi{\left(r \right)} - 1} & 0 & 0 & 0\\ 0 & 0 & 0 & 0\\ 0 & 0 & 0 & 0 \end{pmatrix} & \begin{pmatrix} 0 & 0 & \frac{r \left(- r \frac{d}{d r} \Phi{\left(r \right)} - 2 \Phi{\left(r \right)} + 1\right) \frac{d}{d r} \Phi{\left(r \right)}}{4 \Phi^{2}{\left(r \right)} - 1} & 0\\ 0 & 0 & 0 & 0\\ \frac{r \left(r \frac{d}{d r} \Phi{\left(r \right)} + 2 \Phi{\left(r \right)} - 1\right) \frac{d}{d r} \Phi{\left(r \right)}}{4 \Phi^{2}{\left(r \right)} - 1} & 0 & 0 & 0\\ 0 & 0 & 0 & 0 \end{pmatrix} & \begin{pmatrix} 0 & 0 & 0 & \frac{r \left(- r \frac{d}{d r} \Phi{\left(r \right)} - 2 \Phi{\left(r \right)} + 1\right) \sin^{2}{\left(\theta \right)} \frac{d}{d r} \Phi{\left(r \right)}}{4 \Phi^{2}{\left(r \right)} - 1}\\ 0 & 0 & 0 & 0\\ 0 & 0 & 0 & 0\\ \frac{r \left(r \frac{d}{d r} \Phi{\left(r \right)} + 2 \Phi{\left(r \right)} - 1\right) \sin^{2}{\left(\theta \right)} \frac{d}{d r} \Phi{\left(r \right)}}{4 \Phi^{2}{\left(r \right)} - 1} & 0 & 0 & 0 \end{pmatrix}\\ \begin{pmatrix} 0 & \frac{4 \Phi^{2}{\left(r \right)} \frac{d^{2}}{d r^{2}} \Phi{\left(r \right)} - 4 \Phi{\left(r \right)} \left(\frac{d}{d r} \Phi{\left(r \right)}\right)^{2} - \frac{d^{2}}{d r^{2}} \Phi{\left(r \right)}}{8 \Phi^{3}{\left(r \right)} - 4 \Phi^{2}{\left(r \right)} - 2 \Phi{\left(r \right)} + 1} & 0 & 0\\ \frac{- 4 \Phi^{2}{\left(r \right)} \frac{d^{2}}{d r^{2}} \Phi{\left(r \right)} + 4 \Phi{\left(r \right)} \left(\frac{d}{d r} \Phi{\left(r \right)}\right)^{2} + \frac{d^{2}}{d r^{2}} \Phi{\left(r \right)}}{8 \Phi^{3}{\left(r \right)} - 4 \Phi^{2}{\left(r \right)} - 2 \Phi{\left(r \right)} + 1} & 0 & 0 & 0\\ 0 & 0 & 0 & 0\\ 0 & 0 & 0 & 0 \end{pmatrix} & \begin{pmatrix} 0 & 0 & 0 & 0\\ 0 & 0 & 0 & 0\\ 0 & 0 & 0 & 0\\ 0 & 0 & 0 & 0 \end{pmatrix} & \begin{pmatrix} 0 & 0 & 0 & 0\\ 0 & 0 & \frac{r \left(- 2 r \Phi{\left(r \right)} \frac{d^{2}}{d r^{2}} \Phi{\left(r \right)} + 2 r \left(\frac{d}{d r} \Phi{\left(r \right)}\right)^{2} + r \frac{d^{2}}{d r^{2}} \Phi{\left(r \right)} - 2 \Phi{\left(r \right)} \frac{d}{d r} \Phi{\left(r \right)} + \frac{d}{d r} \Phi{\left(r \right)}\right)}{4 \Phi^{2}{\left(r \right)} - 4 \Phi{\left(r \right)} + 1} & 0\\ 0 & \frac{r \left(2 r \Phi{\left(r \right)} \frac{d^{2}}{d r^{2}} \Phi{\left(r \right)} - 2 r \left(\frac{d}{d r} \Phi{\left(r \right)}\right)^{2} - r \frac{d^{2}}{d r^{2}} \Phi{\left(r \right)} + 2 \Phi{\left(r \right)} \frac{d}{d r} \Phi{\left(r \right)} - \frac{d}{d r} \Phi{\left(r \right)}\right)}{4 \Phi^{2}{\left(r \right)} - 4 \Phi{\left(r \right)} + 1} & 0 & 0\\ 0 & 0 & 0 & 0 \end{pmatrix} & \begin{pmatrix} 0 & 0 & 0 & 0\\ 0 & 0 & 0 & \frac{r \left(- 2 r \Phi{\left(r \right)} \frac{d^{2}}{d r^{2}} \Phi{\left(r \right)} + 2 r \left(\frac{d}{d r} \Phi{\left(r \right)}\right)^{2} + r \frac{d^{2}}{d r^{2}} \Phi{\left(r \right)} - 2 \Phi{\left(r \right)} \frac{d}{d r} \Phi{\left(r \right)} + \frac{d}{d r} \Phi{\left(r \right)}\right) \sin^{2}{\left(\theta \right)}}{4 \Phi^{2}{\left(r \right)} - 4 \Phi{\left(r \right)} + 1}\\ 0 & 0 & 0 & 0\\ 0 & \frac{r \left(2 r \Phi{\left(r \right)} \frac{d^{2}}{d r^{2}} \Phi{\left(r \right)} - 2 r \left(\frac{d}{d r} \Phi{\left(r \right)}\right)^{2} - r \frac{d^{2}}{d r^{2}} \Phi{\left(r \right)} + 2 \Phi{\left(r \right)} \frac{d}{d r} \Phi{\left(r \right)} - \frac{d}{d r} \Phi{\left(r \right)}\right) \sin^{2}{\left(\theta \right)}}{4 \Phi^{2}{\left(r \right)} - 4 \Phi{\left(r \right)} + 1} & 0 & 0 \end{pmatrix}\\ \begin{pmatrix} 0 & 0 & \frac{\left(r \frac{d}{d r} \Phi{\left(r \right)} + 2 \Phi{\left(r \right)} - 1\right) \frac{d}{d r} \Phi{\left(r \right)}}{r \left(2 \Phi{\left(r \right)} - 1\right)^{2}} & 0\\ 0 & 0 & 0 & 0\\ - \frac{\left(r \frac{d}{d r} \Phi{\left(r \right)} + 2 \Phi{\left(r \right)} - 1\right) \frac{d}{d r} \Phi{\left(r \right)}}{r \left(2 \Phi{\left(r \right)} - 1\right)^{2}} & 0 & 0 & 0\\ 0 & 0 & 0 & 0 \end{pmatrix} & \begin{pmatrix} 0 & 0 & 0 & 0\\ 0 & 0 & \frac{2 r \Phi{\left(r \right)} \frac{d^{2}}{d r^{2}} \Phi{\left(r \right)} - 2 r \left(\frac{d}{d r} \Phi{\left(r \right)}\right)^{2} - r \frac{d^{2}}{d r^{2}} \Phi{\left(r \right)} + 2 \Phi{\left(r \right)} \frac{d}{d r} \Phi{\left(r \right)} - \frac{d}{d r} \Phi{\left(r \right)}}{r \left(4 \Phi^{2}{\left(r \right)} - 4 \Phi{\left(r \right)} + 1\right)} & 0\\ 0 & \frac{- 2 r \Phi{\left(r \right)} \frac{d^{2}}{d r^{2}} \Phi{\left(r \right)} + 2 r \left(\frac{d}{d r} \Phi{\left(r \right)}\right)^{2} + r \frac{d^{2}}{d r^{2}} \Phi{\left(r \right)} - 2 \Phi{\left(r \right)} \frac{d}{d r} \Phi{\left(r \right)} + \frac{d}{d r} \Phi{\left(r \right)}}{r \left(4 \Phi^{2}{\left(r \right)} - 4 \Phi{\left(r \right)} + 1\right)} & 0 & 0\\ 0 & 0 & 0 & 0 \end{pmatrix} & \begin{pmatrix} 0 & 0 & 0 & 0\\ 0 & 0 & 0 & 0\\ 0 & 0 & 0 & 0\\ 0 & 0 & 0 & 0 \end{pmatrix} & \begin{pmatrix} 0 & 0 & 0 & 0\\ 0 & 0 & 0 & 0\\ 0 & 0 & 0 & \sin^{2}{\left(\theta \right)} - \frac{\left(r \frac{d}{d r} \Phi{\left(r \right)} + 2 \Phi{\left(r \right)} - 1\right)^{2} \sin^{2}{\left(\theta \right)}}{\left(2 \Phi{\left(r \right)} - 1\right)^{2}}\\ 0 & 0 & - \sin^{2}{\left(\theta \right)} + \frac{\left(r \frac{d}{d r} \Phi{\left(r \right)} + 2 \Phi{\left(r \right)} - 1\right)^{2} \sin^{2}{\left(\theta \right)}}{\left(2 \Phi{\left(r \right)} - 1\right)^{2}} & 0 \end{pmatrix}\\ \begin{pmatrix} 0 & 0 & 0 & \frac{\left(r \frac{d}{d r} \Phi{\left(r \right)} + 2 \Phi{\left(r \right)} - 1\right) \frac{d}{d r} \Phi{\left(r \right)}}{r \left(2 \Phi{\left(r \right)} - 1\right)^{2}}\\ 0 & 0 & 0 & 0\\ 0 & 0 & 0 & 0\\ - \frac{\left(r \frac{d}{d r} \Phi{\left(r \right)} + 2 \Phi{\left(r \right)} - 1\right) \frac{d}{d r} \Phi{\left(r \right)}}{r \left(2 \Phi{\left(r \right)} - 1\right)^{2}} & 0 & 0 & 0 \end{pmatrix} & \begin{pmatrix} 0 & 0 & 0 & 0\\ 0 & 0 & 0 & \frac{2 r \Phi{\left(r \right)} \frac{d^{2}}{d r^{2}} \Phi{\left(r \right)} - 2 r \left(\frac{d}{d r} \Phi{\left(r \right)}\right)^{2} - r \frac{d^{2}}{d r^{2}} \Phi{\left(r \right)} + 2 \Phi{\left(r \right)} \frac{d}{d r} \Phi{\left(r \right)} - \frac{d}{d r} \Phi{\left(r \right)}}{r \left(4 \Phi^{2}{\left(r \right)} - 4 \Phi{\left(r \right)} + 1\right)}\\ 0 & 0 & 0 & 0\\ 0 & \frac{- 2 r \Phi{\left(r \right)} \frac{d^{2}}{d r^{2}} \Phi{\left(r \right)} + 2 r \left(\frac{d}{d r} \Phi{\left(r \right)}\right)^{2} + r \frac{d^{2}}{d r^{2}} \Phi{\left(r \right)} - 2 \Phi{\left(r \right)} \frac{d}{d r} \Phi{\left(r \right)} + \frac{d}{d r} \Phi{\left(r \right)}}{r \left(4 \Phi^{2}{\left(r \right)} - 4 \Phi{\left(r \right)} + 1\right)} & 0 & 0 \end{pmatrix} & \begin{pmatrix} 0 & 0 & 0 & 0\\ 0 & 0 & 0 & 0\\ 0 & 0 & 0 & \frac{r \left(r \frac{d}{d r} \Phi{\left(r \right)} + 4 \Phi{\left(r \right)} - 2\right) \frac{d}{d r} \Phi{\left(r \right)}}{4 \Phi^{2}{\left(r \right)} - 4 \Phi{\left(r \right)} + 1}\\ 0 & 0 & \frac{r \left(- r \frac{d}{d r} \Phi{\left(r \right)} - 4 \Phi{\left(r \right)} + 2\right) \frac{d}{d r} \Phi{\left(r \right)}}{4 \Phi^{2}{\left(r \right)} - 4 \Phi{\left(r \right)} + 1} & 0 \end{pmatrix} & \begin{pmatrix} 0 & 0 & 0 & 0\\ 0 & 0 & 0 & 0\\ 0 & 0 & 0 & 0\\ 0 & 0 & 0 & 0 \end{pmatrix} \end{pmatrix},$$
$$R_{ab}=\begin{pmatrix} \frac{- 4 r \Phi^{2}{\left(r \right)} \frac{d^{2}}{d r^{2}} \Phi{\left(r \right)} - 2 r \left(\frac{d}{d r} \Phi{\left(r \right)}\right)^{2} + r \frac{d^{2}}{d r^{2}} \Phi{\left(r \right)} - 8 \Phi^{2}{\left(r \right)} \frac{d}{d r} \Phi{\left(r \right)} + 2 \frac{d}{d r} \Phi{\left(r \right)}}{r \left(8 \Phi^{3}{\left(r \right)} - 4 \Phi^{2}{\left(r \right)} - 2 \Phi{\left(r \right)} + 1\right)} & 0 & 0 & 0\\ 0 & \frac{- 24 r \Phi^{3}{\left(r \right)} \frac{d^{2}}{d r^{2}} \Phi{\left(r \right)} + 24 r \Phi^{2}{\left(r \right)} \left(\frac{d}{d r} \Phi{\left(r \right)}\right)^{2} - 4 r \Phi^{2}{\left(r \right)} \frac{d^{2}}{d r^{2}} \Phi{\left(r \right)} + 12 r \Phi{\left(r \right)} \left(\frac{d}{d r} \Phi{\left(r \right)}\right)^{2} + 6 r \Phi{\left(r \right)} \frac{d^{2}}{d r^{2}} \Phi{\left(r \right)} + 4 r \left(\frac{d}{d r} \Phi{\left(r \right)}\right)^{2} + r \frac{d^{2}}{d r^{2}} \Phi{\left(r \right)} - 16 \Phi^{3}{\left(r \right)} \frac{d}{d r} \Phi{\left(r \right)} - 8 \Phi^{2}{\left(r \right)} \frac{d}{d r} \Phi{\left(r \right)} + 4 \Phi{\left(r \right)} \frac{d}{d r} \Phi{\left(r \right)} + 2 \frac{d}{d r} \Phi{\left(r \right)}}{r \left(16 \Phi^{4}{\left(r \right)} - 8 \Phi^{2}{\left(r \right)} + 1\right)} & 0 & 0\\ 0 & 0 & \frac{r \left(- 4 r \Phi^{2}{\left(r \right)} \frac{d^{2}}{d r^{2}} \Phi{\left(r \right)} + 2 r \left(\frac{d}{d r} \Phi{\left(r \right)}\right)^{2} + r \frac{d^{2}}{d r^{2}} \Phi{\left(r \right)} - 16 \Phi^{2}{\left(r \right)} \frac{d}{d r} \Phi{\left(r \right)} + 4 \Phi{\left(r \right)} \frac{d}{d r} \Phi{\left(r \right)} + 2 \frac{d}{d r} \Phi{\left(r \right)}\right)}{8 \Phi^{3}{\left(r \right)} - 4 \Phi^{2}{\left(r \right)} - 2 \Phi{\left(r \right)} + 1} & 0\\ 0 & 0 & 0 & \frac{r \left(- 4 r \Phi^{2}{\left(r \right)} \frac{d^{2}}{d r^{2}} \Phi{\left(r \right)} + 2 r \left(\frac{d}{d r} \Phi{\left(r \right)}\right)^{2} + r \frac{d^{2}}{d r^{2}} \Phi{\left(r \right)} - 16 \Phi^{2}{\left(r \right)} \frac{d}{d r} \Phi{\left(r \right)} + 4 \Phi{\left(r \right)} \frac{d}{d r} \Phi{\left(r \right)} + 2 \frac{d}{d r} \Phi{\left(r \right)}\right) \sin^{2}{\left(\theta \right)}}{8 \Phi^{3}{\left(r \right)} - 4 \Phi^{2}{\left(r \right)} - 2 \Phi{\left(r \right)} + 1} \end{pmatrix},$$
$$ R=\frac{2 \cdot \left(24 r \Phi^{3}{\left(r \right)} \frac{d^{2}}{d r^{2}} \Phi{\left(r \right)} - 12 r \Phi^{2}{\left(r \right)} \left(\frac{d}{d r} \Phi{\left(r \right)}\right)^{2} + 4 r \Phi^{2}{\left(r \right)} \frac{d^{2}}{d r^{2}} \Phi{\left(r \right)} - 8 r \Phi{\left(r \right)} \left(\frac{d}{d r} \Phi{\left(r \right)}\right)^{2} - 6 r \Phi{\left(r \right)} \frac{d^{2}}{d r^{2}} \Phi{\left(r \right)} - 5 r \left(\frac{d}{d r} \Phi{\left(r \right)}\right)^{2} - r \frac{d^{2}}{d r^{2}} \Phi{\left(r \right)} + 48 \Phi^{3}{\left(r \right)} \frac{d}{d r} \Phi{\left(r \right)} + 8 \Phi^{2}{\left(r \right)} \frac{d}{d r} \Phi{\left(r \right)} - 12 \Phi{\left(r \right)} \frac{d}{d r} \Phi{\left(r \right)} - 2 \frac{d}{d r} \Phi{\left(r \right)}\right)}{r \left(32 \Phi^{5}{\left(r \right)} - 16 \Phi^{4}{\left(r \right)} - 16 \Phi^{3}{\left(r \right)} + 8 \Phi^{2}{\left(r \right)} + 2 \Phi{\left(r \right)} - 1\right)}.$$
\end{landscape}
\newpage %%%%%%%%%%%%%%%%%%%%%%%%%%%%%%%%%%%%%%%%%%%%%%%%%%%%%%%% 

    \hypertarget{friedmannlemauxeetrerobertsonwalker}{%
\paragraph{Friedmann--Lemaître--Robertson--Walker}\label{friedmannlemauxeetrerobertsonwalker}}

One of the solution for Einstein field equations, is
Friedmann--Lemaître--Robertson--Walker metric, where the line element is
given by \[
ds^2 = dt^2 - \frac{a^2(t)}{1-kr^2} dr^2 - r^2 a^2(t) d\theta^2 - r^2 a^2(t) \sin^2\theta d\varphi,
\] where \(a\) is a function of time \(a(t)\) (called scale factor),
then we can calculate the \textbf{\emph{Christoffel symbols}},
\textbf{\emph{Riemann tensor}}, \textbf{\emph{Ricci tensor}} and
\textbf{\emph{Ricci scalar}}.

\begin{quote}
Here using Geometrized unit system \(G=c=1\).
\end{quote}

    \begin{tcolorbox}[breakable, size=fbox, boxrule=1pt, pad at break*=1mm,colback=cellbackground, colframe=cellborder]
\prompt{In}{incolor}{14}{\boxspacing}
\begin{Verbatim}[commandchars=\\\{\}]
\PY{c+c1}{\PYZsh{} coordinates }
\PY{n}{t}\PY{p}{,} \PY{n}{r}\PY{p}{,} \PY{n}{theta}\PY{p}{,} \PY{n}{phi} \PY{o}{=} \PY{n}{sp}\PY{o}{.}\PY{n}{symbols}\PY{p}{(}\PY{l+s+s1}{\PYZsq{}}\PY{l+s+s1}{t r theta varphi}\PY{l+s+s1}{\PYZsq{}}\PY{p}{)}
\PY{c+c1}{\PYZsh{} constants}
\PY{n}{k}\PY{p}{,} \PY{n}{c} \PY{o}{=} \PY{n}{sp}\PY{o}{.}\PY{n}{symbols}\PY{p}{(}\PY{l+s+s1}{\PYZsq{}}\PY{l+s+s1}{k c}\PY{l+s+s1}{\PYZsq{}}\PY{p}{)}
\PY{c+c1}{\PYZsh{} dependent function}
\PY{n}{a} \PY{o}{=} \PY{n}{sp}\PY{o}{.}\PY{n}{Function}\PY{p}{(}\PY{l+s+s1}{\PYZsq{}}\PY{l+s+s1}{a}\PY{l+s+s1}{\PYZsq{}}\PY{p}{)}\PY{p}{(}\PY{n}{t}\PY{p}{)}
\PY{c+c1}{\PYZsh{} \PYZhy{}\PYZhy{}\PYZhy{}\PYZhy{}\PYZhy{}\PYZhy{}\PYZhy{}\PYZhy{}\PYZhy{}\PYZhy{}\PYZhy{}\PYZhy{}\PYZhy{}\PYZhy{}\PYZhy{}\PYZhy{}\PYZhy{}\PYZhy{}\PYZhy{}\PYZhy{}\PYZhy{}\PYZhy{}\PYZhy{}\PYZhy{}\PYZhy{}\PYZhy{}\PYZhy{}\PYZhy{}\PYZhy{}\PYZhy{}}
\PY{n}{coords} \PY{o}{=} \PY{n}{sp}\PY{o}{.}\PY{n}{Matrix}\PY{p}{(}\PY{p}{[}\PY{n}{t}\PY{p}{,} \PY{n}{r}\PY{p}{,} \PY{n}{theta}\PY{p}{,} \PY{n}{phi}\PY{p}{]}\PY{p}{)}
\PY{n}{metric} \PY{o}{=} \PY{n}{sp}\PY{o}{.}\PY{n}{diag}\PY{p}{(}\PY{l+m+mi}{1}\PY{p}{,} \PY{o}{\PYZhy{}}\PY{n}{a}\PY{o}{*}\PY{o}{*}\PY{l+m+mi}{2}\PY{o}{/}\PY{p}{(}\PY{l+m+mi}{1}\PY{o}{\PYZhy{}}\PY{n}{k}\PY{o}{*}\PY{n}{r}\PY{o}{*}\PY{o}{*}\PY{l+m+mi}{2}\PY{p}{)}\PY{p}{,} \PY{o}{\PYZhy{}}\PY{n}{r}\PY{o}{*}\PY{o}{*}\PY{l+m+mi}{2}\PY{o}{*}\PY{n}{a}\PY{o}{*}\PY{o}{*}\PY{l+m+mi}{2}\PY{p}{,} \PY{o}{\PYZhy{}}\PY{n}{r}\PY{o}{*}\PY{o}{*}\PY{l+m+mi}{2}\PY{o}{*}\PY{n}{a}\PY{o}{*}\PY{o}{*}\PY{l+m+mi}{2}\PY{o}{*}\PY{p}{(}\PY{n}{sp}\PY{o}{.}\PY{n}{sin}\PY{p}{(}\PY{n}{theta}\PY{p}{)}\PY{p}{)}\PY{o}{*}\PY{o}{*}\PY{l+m+mi}{2}\PY{p}{)}
\PY{n}{geo} \PY{o}{=} \PY{n}{RiemannGeometry}\PY{p}{(}\PY{n}{metric}\PY{p}{,} \PY{n}{coords}\PY{p}{)}
\PY{n}{geo}\PY{o}{.}\PY{n}{calculate}\PY{p}{(}\PY{p}{)}
\PY{n}{geo}
\end{Verbatim}
\end{tcolorbox}

    \begin{Verbatim}[commandchars=\\\{\}]
Calculating {\ldots}
    \end{Verbatim}

    \begin{Verbatim}[commandchars=\\\{\}]
Christoffel Symbols : 100\%|\textblock\textblock\textblock\textblock\textblock\textblock\textblock\textblock\textblock\textblock\textblock\textblock\textblock\textblock\textblock\textblock\textblock\textblock\textblock\textblock\textblock\textblock\textblock\textblock\textblock\textblock\textblock\textblock\textblock\textblock\textblock\textblock\textblock\textblock\textblock| 4/4[00:00<00:00, 31.66it/s]
Riemann Tensor      : 100\%|\textblock\textblock\textblock\textblock\textblock\textblock\textblock\textblock\textblock\textblock\textblock\textblock\textblock\textblock\textblock\textblock\textblock\textblock\textblock\textblock\textblock\textblock\textblock\textblock\textblock\textblock\textblock\textblock\textblock\textblock\textblock\textblock\textblock\textblock\textblock| 4/4[00:00<00:00,  9.94it/s]
Ricci Tensor        : 100\%|\textblock\textblock\textblock\textblock\textblock\textblock\textblock\textblock\textblock\textblock\textblock\textblock\textblock\textblock\textblock\textblock\textblock\textblock\textblock\textblock\textblock\textblock\textblock\textblock\textblock\textblock\textblock\textblock\textblock\textblock\textblock\textblock\textblock\textblock\textblock| 4/4[00:00<00:00, 51.95it/s]
Ricci Scalar        : 100\%|\textblock\textblock\textblock\textblock\textblock\textblock\textblock\textblock\textblock\textblock\textblock\textblock\textblock\textblock\textblock\textblock\textblock\textblock\textblock\textblock\textblock\textblock\textblock\textblock\textblock\textblock\textblock\textblock\textblock\textblock\textblock\textblock| 4/4 [00:00<00:00,
13740.55it/s]
    \end{Verbatim}
 
            
\prompt{Out}{outcolor}{14}{}
    
    $$x^{a}=\begin{pmatrix}t\\r\\\theta\\\varphi\end{pmatrix},\quad g_{ab}=\begin{pmatrix} 1 & 0 & 0 & 0\\ 0 & - \frac{a^{2}{\left(t \right)}}{- k r^{2} + 1} & 0 & 0\\ 0 & 0 & - r^{2} a^{2}{\left(t \right)} & 0\\ 0 & 0 & 0 & - r^{2} a^{2}{\left(t \right)} \sin^{2}{\left(\theta \right)} \end{pmatrix},\quad g^{ab}=\begin{pmatrix} 1 & 0 & 0 & 0\\ 0 & \frac{k r^{2} - 1}{a^{2}{\left(t \right)}} & 0 & 0\\ 0 & 0 & - \frac{1}{r^{2} a^{2}{\left(t \right)}} & 0\\ 0 & 0 & 0 & - \frac{1}{r^{2} a^{2}{\left(t \right)} \sin^{2}{\left(\theta \right)}} \end{pmatrix},$$
$${\Gamma^{a}}_{bc}=\begin{pmatrix} \begin{pmatrix} 0 & 0 & 0 & 0\\ 0 & - \frac{a{\left(t \right)} \frac{d}{d t} a{\left(t \right)}}{k r^{2} - 1} & 0 & 0\\ 0 & 0 & r^{2} a{\left(t \right)} \frac{d}{d t} a{\left(t \right)} & 0\\ 0 & 0 & 0 & r^{2} a{\left(t \right)} \sin^{2}{\left(\theta \right)} \frac{d}{d t} a{\left(t \right)} \end{pmatrix} & \begin{pmatrix} 0 & \frac{\frac{d}{d t} a{\left(t \right)}}{a{\left(t \right)}} & 0 & 0\\ \frac{\frac{d}{d t} a{\left(t \right)}}{a{\left(t \right)}} & - \frac{k r}{k r^{2} - 1} & 0 & 0\\ 0 & 0 & k r^{3} - r & 0\\ 0 & 0 & 0 & r \left(k r^{2} - 1\right) \sin^{2}{\left(\theta \right)} \end{pmatrix} & \begin{pmatrix} 0 & 0 & \frac{\frac{d}{d t} a{\left(t \right)}}{a{\left(t \right)}} & 0\\ 0 & 0 & \frac{1}{r} & 0\\ \frac{\frac{d}{d t} a{\left(t \right)}}{a{\left(t \right)}} & \frac{1}{r} & 0 & 0\\ 0 & 0 & 0 & - \sin{\left(\theta \right)} \cos{\left(\theta \right)} \end{pmatrix} & \begin{pmatrix} 0 & 0 & 0 & \frac{\frac{d}{d t} a{\left(t \right)}}{a{\left(t \right)}}\\ 0 & 0 & 0 & \frac{1}{r}\\ 0 & 0 & 0 & \frac{1}{\tan{\left(\theta \right)}}\\ \frac{\frac{d}{d t} a{\left(t \right)}}{a{\left(t \right)}} & \frac{1}{r} & \frac{1}{\tan{\left(\theta \right)}} & 0 \end{pmatrix} \end{pmatrix},$$
$${R^{a}}_{bcd}=\begin{pmatrix} \begin{pmatrix} 0 & 0 & 0 & 0\\ 0 & 0 & 0 & 0\\ 0 & 0 & 0 & 0\\ 0 & 0 & 0 & 0 \end{pmatrix} & \begin{pmatrix} 0 & - \frac{a{\left(t \right)} \frac{d^{2}}{d t^{2}} a{\left(t \right)}}{k r^{2} - 1} & 0 & 0\\ \frac{a{\left(t \right)} \frac{d^{2}}{d t^{2}} a{\left(t \right)}}{k r^{2} - 1} & 0 & 0 & 0\\ 0 & 0 & 0 & 0\\ 0 & 0 & 0 & 0 \end{pmatrix} & \begin{pmatrix} 0 & 0 & r^{2} a{\left(t \right)} \frac{d^{2}}{d t^{2}} a{\left(t \right)} & 0\\ 0 & 0 & 0 & 0\\ - r^{2} a{\left(t \right)} \frac{d^{2}}{d t^{2}} a{\left(t \right)} & 0 & 0 & 0\\ 0 & 0 & 0 & 0 \end{pmatrix} & \begin{pmatrix} 0 & 0 & 0 & r^{2} a{\left(t \right)} \sin^{2}{\left(\theta \right)} \frac{d^{2}}{d t^{2}} a{\left(t \right)}\\ 0 & 0 & 0 & 0\\ 0 & 0 & 0 & 0\\ - r^{2} a{\left(t \right)} \sin^{2}{\left(\theta \right)} \frac{d^{2}}{d t^{2}} a{\left(t \right)} & 0 & 0 & 0 \end{pmatrix}\\ \begin{pmatrix} 0 & \frac{\frac{d^{2}}{d t^{2}} a{\left(t \right)}}{a{\left(t \right)}} & 0 & 0\\ - \frac{\frac{d^{2}}{d t^{2}} a{\left(t \right)}}{a{\left(t \right)}} & 0 & 0 & 0\\ 0 & 0 & 0 & 0\\ 0 & 0 & 0 & 0 \end{pmatrix} & \begin{pmatrix} 0 & 0 & 0 & 0\\ 0 & 0 & 0 & 0\\ 0 & 0 & 0 & 0\\ 0 & 0 & 0 & 0 \end{pmatrix} & \begin{pmatrix} 0 & 0 & 0 & 0\\ 0 & 0 & r^{2} \left(k + \left(\frac{d}{d t} a{\left(t \right)}\right)^{2}\right) & 0\\ 0 & r^{2} \left(- k - \left(\frac{d}{d t} a{\left(t \right)}\right)^{2}\right) & 0 & 0\\ 0 & 0 & 0 & 0 \end{pmatrix} & \begin{pmatrix} 0 & 0 & 0 & 0\\ 0 & 0 & 0 & r^{2} \left(k + \left(\frac{d}{d t} a{\left(t \right)}\right)^{2}\right) \sin^{2}{\left(\theta \right)}\\ 0 & 0 & 0 & 0\\ 0 & r^{2} \left(- k - \left(\frac{d}{d t} a{\left(t \right)}\right)^{2}\right) \sin^{2}{\left(\theta \right)} & 0 & 0 \end{pmatrix}\\ \begin{pmatrix} 0 & 0 & \frac{\frac{d^{2}}{d t^{2}} a{\left(t \right)}}{a{\left(t \right)}} & 0\\ 0 & 0 & 0 & 0\\ - \frac{\frac{d^{2}}{d t^{2}} a{\left(t \right)}}{a{\left(t \right)}} & 0 & 0 & 0\\ 0 & 0 & 0 & 0 \end{pmatrix} & \begin{pmatrix} 0 & 0 & 0 & 0\\ 0 & 0 & \frac{k + \left(\frac{d}{d t} a{\left(t \right)}\right)^{2}}{k r^{2} - 1} & 0\\ 0 & \frac{- k - \left(\frac{d}{d t} a{\left(t \right)}\right)^{2}}{k r^{2} - 1} & 0 & 0\\ 0 & 0 & 0 & 0 \end{pmatrix} & \begin{pmatrix} 0 & 0 & 0 & 0\\ 0 & 0 & 0 & 0\\ 0 & 0 & 0 & 0\\ 0 & 0 & 0 & 0 \end{pmatrix} & \begin{pmatrix} 0 & 0 & 0 & 0\\ 0 & 0 & 0 & 0\\ 0 & 0 & 0 & r^{2} \left(k + \left(\frac{d}{d t} a{\left(t \right)}\right)^{2}\right) \sin^{2}{\left(\theta \right)}\\ 0 & 0 & r^{2} \left(- k - \left(\frac{d}{d t} a{\left(t \right)}\right)^{2}\right) \sin^{2}{\left(\theta \right)} & 0 \end{pmatrix}\\ \begin{pmatrix} 0 & 0 & 0 & \frac{\frac{d^{2}}{d t^{2}} a{\left(t \right)}}{a{\left(t \right)}}\\ 0 & 0 & 0 & 0\\ 0 & 0 & 0 & 0\\ - \frac{\frac{d^{2}}{d t^{2}} a{\left(t \right)}}{a{\left(t \right)}} & 0 & 0 & 0 \end{pmatrix} & \begin{pmatrix} 0 & 0 & 0 & 0\\ 0 & 0 & 0 & \frac{k + \left(\frac{d}{d t} a{\left(t \right)}\right)^{2}}{k r^{2} - 1}\\ 0 & 0 & 0 & 0\\ 0 & \frac{- k - \left(\frac{d}{d t} a{\left(t \right)}\right)^{2}}{k r^{2} - 1} & 0 & 0 \end{pmatrix} & \begin{pmatrix} 0 & 0 & 0 & 0\\ 0 & 0 & 0 & 0\\ 0 & 0 & 0 & r^{2} \left(- k - \left(\frac{d}{d t} a{\left(t \right)}\right)^{2}\right)\\ 0 & 0 & r^{2} \left(k + \left(\frac{d}{d t} a{\left(t \right)}\right)^{2}\right) & 0 \end{pmatrix} & \begin{pmatrix} 0 & 0 & 0 & 0\\ 0 & 0 & 0 & 0\\ 0 & 0 & 0 & 0\\ 0 & 0 & 0 & 0 \end{pmatrix} \end{pmatrix},$$
$$R_{ab}=\begin{pmatrix} - \frac{3 \frac{d^{2}}{d t^{2}} a{\left(t \right)}}{a{\left(t \right)}} & 0 & 0 & 0\\ 0 & \frac{- 2 k - a{\left(t \right)} \frac{d^{2}}{d t^{2}} a{\left(t \right)} - 2 \left(\frac{d}{d t} a{\left(t \right)}\right)^{2}}{k r^{2} - 1} & 0 & 0\\ 0 & 0 & r^{2} \cdot \left(2 k + a{\left(t \right)} \frac{d^{2}}{d t^{2}} a{\left(t \right)} + 2 \left(\frac{d}{d t} a{\left(t \right)}\right)^{2}\right) & 0\\ 0 & 0 & 0 & r^{2} \cdot \left(2 k + a{\left(t \right)} \frac{d^{2}}{d t^{2}} a{\left(t \right)} + 2 \left(\frac{d}{d t} a{\left(t \right)}\right)^{2}\right) \sin^{2}{\left(\theta \right)} \end{pmatrix},\quad R=\frac{6 \left(- k - a{\left(t \right)} \frac{d^{2}}{d t^{2}} a{\left(t \right)} - \left(\frac{d}{d t} a{\left(t \right)}\right)^{2}\right)}{a^{2}{\left(t \right)}}.$$

    


    % Add a bibliography block to the postdoc
    
    
    
\end{document}
